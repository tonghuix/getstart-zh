%%
%% getstart.tex -- Flight Gear documentation: The FlightGear Manual
%% Chapter file
%%
%% Copyright (C) 2002 Michael Basler
%%                  & Bernhard Buckel
%%
%% This program is free software; you can redistribute it and/or
%% modify it under the terms of the GNU General Public License as
%% published by the Free Software Foundation; either version 2 of the
%% License, or (at your option) any later version.
%%
%% This program is distributed in the hope that it will be useful, but
%% WITHOUT ANY WARRANTY; without even the implied warranty of
%% MERCHANTABILITY or FITNESS FOR A PARTICULAR PURPOSE.  See the GNU
%% General Public License for more details.
%%
%% You should have received a copy of the GNU General Public License
%% along with this program; if not, write to the Free Software
%% Foundation, Inc., 675 Mass Ave, Cambridge, MA 02139, USA.
%%
%% $Id: takeof.tex,v 0.6 2002/09/09 michael
%% (Log is kept at end of this file)

%%%%%%%%%%%%%%%%%%%%%%%%%%%%%%%%%%%%%%%%%%%%%%%%%%%%%%%%%%%%%%%%%%%%%%%%%%%%%%%%%%%%%%%%%%%%%%%
\ifchinese
\chapter{{\\}起飞:如何启动程序}
\fi
%\IfLanguageName{english}{
%\chapter{Takeoff: How to start the program}
%}{}
\IfLanguageName{french}{
\chapter{D\'{e}collage : comment d\'{e}marrer le programme}
}{}
\IfLanguageName{italian}{
\chapter{Inizio: come avviare il programma}
}{}
\label{takeoff}
%%%%%%%%%%%%%%%%%%%%%%%%%%%%%%%%%%%%%%%%%%%%%%%%%%%%%%%%%%%%%%%%%%%%%%%%%%%%%%%%%%%%%%%%%%%%%%%
\ifchinese
\markboth{\thechapter.\hspace*{1mm} 起飞}{\thesection\hspace*{1mm} 命令行参数}
\fi
%\IfLanguageName{english}{
%\markboth{\thechapter.\hspace*{1mm} TAKEOFF}{\thesection\hspace*{1mm} Command line parameters}
%}{}
\IfLanguageName{french}{
\markboth{\thechapter.\hspace*{1mm} DECOLLAGE}{\thesection\hspace*{1mm} Param\`{e}tres de ligne de commande }
}{}
\IfLanguageName{italian}{
\markboth{\thechapter.\hspace*{1mm} INIZIO}{\thesection\hspace*{1mm} Come avviare il programma }
}{}

%%%%%%%%%%%%%%%%%%%%%%%%%%%%%%%%%%%%%%%%%%%%%%%%%%%%%%%%%%%%%%%%%%%%%%%%%%%%%%%%%%%%%%%%%%%%%%%
\ifchinese
\section{启动模拟器}\index{启动模拟器}\index{启动 FlightGear}
\fi
%\IfLanguageName{english}{
%\section{Environment Variables}\index{environment variables}
%}{}
\IfLanguageName{french}{
\section{Variables d'environnement}\index{variables d'environment}
}{}
\IfLanguageName{italian}{
\section{Variabili d'ambiente}\index{Variabili d'ambiente}
}{}
%%%%%%%%%%%%%%%%%%%%%%%%%%%%%%%%%%%%%%%%%%%%%%%%%%%%%%%%%%%%%%%%%%%%%%%%%%%%%%%%%%%%%%%%%%%%%%%

\centerline{\fbox{
\includegraphics[clip,width=15cm]{img/phnl_takeoff_ready}
}}
\smallskip
\noindent
\ifchinese
图\ 3:准备起飞:\textit{在檀香山国际机场(KSFO)的默认启动点等待起飞}
}{}
\fi
\iffalse
\IfLanguageName{french}{
Fig.\,3 : Par\'{e} \`{a} d\'{e}coller : \textit{en attente \`{a} la position de d\'{e}marrage par d\'{e}faut de l'a\'{e}roport de San Francisco Intl., KSFO.}
}{}
\IfLanguageName{italian}{
Fig.\,3 : Pronti al decollo: \textit{avvio predefinito a San Francisco Intl, KSFO.}
}{}
\fi
\medskip

\ifchinese
大多数 FlightGear 的发布版本集成了完整的 FlightGear 启动器以便启动 FlightGear。只需要在开始菜单双击 \texttt{FlightGear Launcher},或者桌面图标即可,还可以在命令行执行 \texttt{fgfs -$ $-launcher}。启动器可以让你方便的选择航空器、起始位置(你甚至可以从距跑道 10 英里开外的地方开始进近,或者穿过一个特定的导航点)、时刻、启用或禁用 TerraSync 或实时天气,以及大量的其他设定。

\medskip
第一次打开启动器的时候,你会看到对话框要求设置你的 \texttt{FG\_ROOT} 变量,正常情况下应该会是

\texttt{c:\char`\\Program Files\char`\\{}FlightGear\char`\\{}data} 或者

\texttt{c:\char`\\Program Files\char`\\FlightGear 2018.3.0\char`\\data}


设置好以后你会看到如下图这样的界面:

\medskip
\centerline{\fbox{
\includegraphics[clip,width=12.5cm]{img/launcher_summary.png}
}}
\smallskip
\noindent
图 4:概览:\textit{启动器设置概览。按“Fly!”按钮来启动模拟器。}
\medskip

启动器默认会使用塞斯纳 172P 在檀香山国际机场的跑到上。只要按\button{Fly!}按钮即可启动模拟器。

或者如果你想改变设置,可以从左侧按钮里修改设置。

\medskip
\centerline{\fbox{
\includegraphics[clip,width=12.5cm]{img/launcher_aircraft.png}
}}
\smallskip
\noindent
图 5:航空器选择:\textit{从大量的可选航空器里选择并自动下载。}
\medskip

你可以点选窗口左侧的“Aircraft”(航空器)按钮来改变你的航空器。

FlightGear 默认会预装塞斯纳 172P 和 UFO。你可以在列表里按“\button{Install}”按钮下载并安装其他任何航空器。你也可以从官方网站或私人源下载航空器。

\medskip
\centerline{\fbox{
\includegraphics[clip,width=12.5cm]{img/launcher_location.png}
}}
\smallskip
\noindent
图 6:起始位置:\textit{选择起始位置在地面或者在空中。}
\medskip

\command{Location}(位置)可以让你选择起始位置——停泊在停机位,放在跑到上准备起飞,或者放在 ILS 近进,又或者相对于 VOR、导航点的位置。默认情况下只会显示当前选定的位置。要选择一个完全不同的位置,按\button{Back}返回按钮并输入一个全世界你想去的位置的名字。

\FlightGear{} 会自动下载需要的地景(假设你已经在\command{Settings}设置页面里选择了)

\command{Environment}(环境)选项页让你可以选择一天中的时段、季节以及天气建模。你可以选择使用当前真实天气状况,或者选择一个特定的天气情景比如高压区域,或雷暴天气。

\command{Settings}(设置)选项页让你可以设置模拟器的各种选项,比如多人连线、自动地景下载和图形界面选项。高级选项可以从左侧的\button{Show More}(显示更多)里看到。

最后,\command{Addons}(插件)选项页让你可以从不同的航空器提供者下载航空器。

当你满意了这些设置以后,按\button{Fly!}来启动模拟器。


%%%%%%%%%%%%%%%%%%%%%%%%%%%%%%%%%%%%%%%%%%%%%%%%%%%%%%%%%%%%%%%%%%%%%%%%%%%%%%%%%
\section{从命令行启动}

\fi
\ifchinese
另外,你可以从命令行启动 FlightGear。不过首先需要手动设置 \texttt{FG\_ROOT} 和 \texttt{FG\_SCENERY} 环境变量。你可以根据具体平台和需求用各种方式来设置。
\fi
%\IfLanguageName{english}{
%There are two environment variables that must be defined to run \FlightGear{}.
%These tell \FlightGear{} where to find its data and scenery.
%
%You can set them in a number of ways depending on your platform and requirements.
%}{}
\IfLanguageName{french}{
Il existe deux variables d'environnement qui doivent \^{e}tre d\'{e}finies pour faire fonctionner \FlightGear{}.
Elles indiquent \`{a} \FlightGear{} o\`{u} trouver ses donn\'{e}es et ses sc\`{e}nes.

Vous pouvez les param\'{e}trer de plusieurs fa\c{c}ons en fonction de votre plate-forme et de vos besoins.
}{}

\IfLanguageName{italian}{
Ci sono due variabili d'ambiente che devono essere definite per eseguire FlightGear. Queste dicono a \FlightGear{}
dove trovare i dati e i paesaggi.
\`{e} possibile impostarle in vari modi a seconda della piattaforma e delle proprie esigienze.
}{}

\subsection{FG\_ROOT}\index{FG\_ROOT}

\ifchinese
这里放置 \FlightGear{} 将去哪里寻找数据文件,比如航空器、导航台的位置、机场频率。在你安装 \FlightGear{} 的 \texttt{data} 子目录下。比如:

 \texttt{/usr/local/share/FlightGear/data} or

  \texttt{c:\char`\\Program Files\char`\\FlightGear\char`\\data}.
\fi
%\IfLanguageName{english}{
%This is where \FlightGear{} will find data files such as aircraft, navigational
%beacon locations, airport frequencies. This is the \texttt{data} subdirectory
%of where you installed \FlightGear{}. e.g.
%\texttt{/usr/local/share/FlightGear/data} or
%\texttt{c:$\backslash$Program Files$\backslash$FlightGear$\backslash$data}.
%}{}
\IfLanguageName{french}{
Il s'agit de l'emplacement o\`{u} \FlightGear{} recherchera ses fichiers de donn\'{e}es comme
les a\'{e}ronefs, les emplacements des balises de navigation, les fr\'{e}quences des a\'{e}roports. Il s'agit du
sous-r\'{e}pertoire \texttt{data} de l'emplacement o\`{u} vous avez install\'{e} \FlightGear{}, par exemple :
\texttt{/usr/local/share/FlightGear/data} ou
\texttt{c:$\backslash$Program Files$\backslash$FlightGear$\backslash$data}.
}{}
\IfLanguageName{italian}{
Questa variabile indica a \FlightGear{} dove trovare i file dati come aerei, luoghi di navigazione,
e frequenze aeroportuali. Di default \`{e} una sottodirectory della cartella di installazione di
\FlightGear{}, ad esempio
\texttt{/usr/local/share/FlightGear/data} o
\texttt{c:$\backslash$Program Files$\backslash$FlightGear$\backslash$data}.
}{}

\subsection{FG\_SCENERY}\index{FG\_SCENERY}
\ifchinese
这里告诉 \FlightGear{} 去哪里找地景文件。可以在此按顺序列出要搜索的目录。在 UNIX (包括 Mac OS X)下用“:”分隔,在 Windows 下则用“;”。
\fi
%\IfLanguageName{english}{
%This is where \FlightGear{} will look for scenery files. It consists of a list
%of directories that will be searched in order. The directories are separated
%by ``:'' on Unix and ``;'' on Windows. e.g.
%}{}
\IfLanguageName{french}{
Il s'agit de l'emplacement o\`{u} \FlightGear{} recherchera ses fichiers de sc\`{e}nes. Il s'agit
d'une liste de r\'{e}pertoires qui seront analys\'{e}s de mani\`{e}re s\'{e}quentielle. Les r\'{e}pertoires
sont s\'{e}par\'{e}s par ``:'' sous Unix et ``;'' sous Windows, par exemple :
}{}
\IfLanguageName{italian}{
Questa variabile indica a FlightGear dove cercare i file degli scenari. Si compone di una lista di
directory che saranno controllate nell'ordine in cui sono scritte. Le directory sono separati da
'':'' in Unix e da '';'' su Windows, ecco due esempi:
}{}

\noindent
{\footnotesize{\texttt{/home/joebloggs/WorldScenery:/usr/local/share/FlightGear/data/Scenery}}}

\noindent
\ifchinese
或者
\fi
%\IfLanguageName{english}{
%or
%}{}
\IfLanguageName{french}{
ou
}{}
\IfLanguageName{italian}{
o
}{}

\noindent
{\footnotesize{\texttt{c:\char`\\Program Files\char`\\FlightGear\char`\\data\char`\\Scenery;\\c:\char`\\Program Files\char`\\FlightGear\char`\\data\char`\\WorldScenery}}}.

%%%%%%%%%%%%%%%%%%%%%%%%%%%%%%%%%%%%%%%%%%%%%%%%%%%%%%%%%%%%%%%%%%%%%%%%%%%%%%%%%%%%%%%%%%%%%%%
\ifchinese
\subsection{在 Windows 下启动模拟器}\index{启动 Flightgear!Windows}\index{启动 Flightgear!Windows}
\fi
%\IfLanguageName{english}{
%\section{Launching the simulator under Windows}\index{Launching Flightgear!Windows}\index{Starting Flightgear!Windows}
%}{}
\IfLanguageName{french}{
\section{D\'{e}marrer le simulateur sous Windows}\index{D\'{e}marrer Flightgear!Windows}\index{D\'{e}marrer Flightgear!Windows}
}{}
\IfLanguageName{french}{
\section{Avvio del simulatore sotto Windows}\index{Avvio del simulatore sotto Windows}\index{Avvio del simulatore!Windows}
}{}
%%%%%%%%%%%%%%%%%%%%%%%%%%%%%%%%%%%%%%%%%%%%%%%%%%%%%%%%%%%%%%%%%%%%%%%%%%%%%%%%%%%%%%%%%%%%%%%

\ifchinese
打开一个命令行终端,切换到你安装的二进制目录下(一般是
 \texttt{c:\char`\\Program Files\char`\\FlightGear\char`\\bin\char`\\Win32})
,键入如下命令修改环境变量
\fi
%\IfLanguageName{english}{
%Alternatively, you can run FlightGear from the command line. To do this, you
%need to set up the \texttt{FG\_ROOT} and \texttt{FG\_SCENERY} environment
%variables manually.
%
%Open a command shell, change to the directory where your binary resides
%(typically something like
%\texttt{c:$\backslash$Program Files$\backslash$FlightGear$\backslash$bin$\backslash$Win32}),
%set the environment variables by typing
%}{}
\IfLanguageName{french}{
Alternativement, vous pouvez d\'{e}marrer FlightGear \`{a} partir de la ligne de commande. Pour cela, vous devez param\'{e}trer
les variables d'environnement \texttt{FG\_ROOT} et \texttt{FG\_SCENERY} manuellement.

Ouvrez une invite de commandes, placez-vous dans le r\'{e}pertoire o\`{u} sont positionn\'{e}s vos binaires
(g\'{e}n\'{e}ralement quelque chose comme
\texttt{c:$\backslash$Program Files$\backslash$FlightGear$\backslash$bin$\backslash$Win32}),
puis param\'{e}trez les variables d'environnement en tapant :
}{}

\IfLanguageName{italian}{
In alternativa, \`{e} possibile eseguire FlightGear dalla riga di comando. Per fare questo, \`{e} necessario impostare le variabili
\texttt{FG\_ROOT} e \texttt{FG\_SCENERY} manualmente.

Aprire il prompt dei comandi, passare alla directory in cui risiedono i binari di FlightGear
(normalmente \texttt{c:$\backslash$Program Files$\backslash$FlightGear$\backslash$bin$\backslash$Win32})
E impostare le variabili d'ambiente digitando
}{}

\medskip
\begin{verbatim}
SET FG_HOME="c:\Program Files\FlightGear"
SET FG_ROOT="c:\Program Files\FlightGear\data"
SET FG_SCENERY="c:\Program Files\FlightGear\data\Scenery"
\end{verbatim}
\medskip

\noindent
\ifchinese
并调用 \FlightGear{}(在同一个命令行终端,环境设定只对当前的命令行有效),可以键入
\fi
%\IfLanguageName{english}{
% and invoke \FlightGear{} (within the same Command shell, as environment
% settings are only valid locally within the same shell) via
%}{}
\IfLanguageName{french}{
 et d\'{e}marrez \FlightGear{} (dans la m\^{e}me fen\^{e}tre d'invite de commandes, car les variables d'environnement sont valides
uniquement localement au sein de la m\^{e}me invite de commandes) par l'interm\'{e}diaire de la commande :
}{}
\IfLanguageName{italian}{
e richiamare FlightGear (all'interno dello stesso prompt dei comandi, dato che impostazioni d'ambiente sopra dichiarate
ono valide solo localmente all'interno della stessa shell) tramite
}{}
\medskip

\texttt{fgfs -$ $-option1 -$ $-option2\dots}
\medskip

\ifchinese
命令行选项在~\ref{options}节有详细介绍。当然你也可以用 Windows 文本编辑器(比如记事本)创建一个批处理文件,文件里输入上面这些命令。从最佳的性能考虑,运行 \FlightGear{} 时建议最小化文本输出窗口。
\fi
%\IfLanguageName{english}{
%Command-line options are described in Chapter~\ref{options}.
%Of course, you can create a batch file with a Windows text editor (like notepad)
%using the lines above.
%For maximum performance it is recommended that you to minimize (iconize) the
%text output window while running \FlightGear{}.
%}{}
\IfLanguageName{french}{
Les options de ligne de commande sont d\'{e}crites dans le chapitre~\ref{options}.
Naturellement, vous pouvez cr\'{e}er un fichier \textit{batch} avec un \'{e}diteur
de texte Windows (comme le bloc-notes) contenant les lignes ci-dessus.
Pour obtenir les meilleures performances \`{a} l'ex\'{e}cution, il est recommand\'{e} de
r\'{e}duire (ic\^{o}nifier) la fen\^{e}tre de sortie pendant que \FlightGear{} est en fonctionnement.
}{}
\IfLanguageName{italian}{
Le opzioni della riga di comando sono descritte nel capitolo~\ref{options}.

Naturalmente, \`{e} possibile creare un file batch con un editor di testo di Windows (come notepad)
utilizzando i comandi di cui sopra. Per le massime prestazioni si consiglia di ridurre al minimo
(Iconize) la finestra di output di testo (il prompt) durante l'esecuzione del simulatore.
}{}

%%%%%%%%%%%%%%%%%%%%%%%%%%%%%%%%%%%%%%%%%%%%%%%%%%%%%%%%%%%%%%%%%%%%%%%%%%%%%%%%%%%%%%%%%%%%%%%
\ifchinese
\subsection{在 UNIX/Linux 下启动模拟器}\index{启动 Flightgear!Linux}\index{启动 Flightgear!Linux}
\fi
%\IfLanguageName{english}{
%\section{Launching the simulator under Unix/Linux}\index{Launching Flightgear!Linux}\index{Starting Flightgear!Linux}
%}{}
\IfLanguageName{french}{
\section{D\'{e}marrer le simulateur sous Unix/Linux}\index{D\'{e}marrer Flightgear!Linux}\index{D\'{e}marrer Flightgear!Linux}
}{}
\IfLanguageName{italian}{
\section{Avvio del simulatore in ambiente Unix / Linux}\index{Avvio del simulatore!Linux}\index{Avvio del simulatore in ambiente Unix / Linux}
}{}
%%%%%%%%%%%%%%%%%%%%%%%%%%%%%%%%%%%%%%%%%%%%%%%%%%%%%%%%%%%%%%%%%%%%%%%%%%%%%%%%%%%%%%%%%%%%%%%

\ifchinese
运行 \FlightGear{} 之前,你需要设置一系列环境变量:
\fi
%\IfLanguageName{english}{
%Before you can run \FlightGear{}, you need to set a couple of environment variables:
%}{}
\IfLanguageName{french}{
Avant de pouvoir d\'{e}marrer \FlightGear{}, vous devez d\'{e}finir deux variables d'environnement :
}{}
\IfLanguageName{italian}{
Prima di poter eseguire FlightGear, \`{e} necessario impostare un paio di variabili d'ambiente:
}{}


\begin{itemize}
\ifchinese
\item 你需要添加 \texttt{/usr/local/share/FlightGear/lib} 到你的\\ \texttt{LD\_LIBRARY\_PATH}
\item \texttt{FG\_ROOT} 必须设置为 \FlightGear{} 的数据安装目录。比如\\
\texttt{/usr/local/share/FlightGear/data}。
\item \texttt{FG\_SCENERY} 必须是一个包含地景的目录列表,用";"分隔。这个变量的功效如同系统的 \texttt{PATH} 变量。\\
比如:\texttt{\$FG\_ROOT/Scenery:\$FG\_ROOT/WorldScenery}。
\fi
%\IfLanguageName{english}{
%\item You must add \texttt{/usr/local/share/FlightGear/lib} to your \texttt{LD\_LIBRARY\_PATH}
%\item \texttt{FG\_ROOT} must be set to the data directory of your \FlightGear{} installation. e.g.% \texttt{/usr/local/share/FlightGear/data}.
%\item \texttt{FG\_SCENERY} should be a list of scenery directories, separated by '':''. This works like \texttt{PATH} when searching for scenery.
%e.g. \texttt{\$FG\_ROOT/Scenery:\$FG\_ROOT/WorldScenery}.
%}{}

\IfLanguageName{french}{
\item Vous devez ajouter \texttt{/usr/local/share/FlightGear/lib} \`{a} votre \texttt{LD\_LIBRARY\_PATH}
\item \texttt{FG\_ROOT} doit \^{e}tre param\'{e}tr\'{e} pour pointer vers le r\'{e}pertoire contenant les donn\'{e}es de votre installation de  \FlightGear{}, par exemple : \texttt{/usr/local/share/FlightGear/data}.
\item \texttt{FG\_SCENERY} doit \^{e}tre une liste de r\'{e}pertoires de sc\`{e}nes, s\'{e}par\'{e}s par '':''. Ce fonctionnement est semblable \`{a} celui de \texttt{PATH} lorsqu'on recherche des sc\`{e}nes.
par exemple : \texttt{\$FG\_ROOT/Scenery:\$FG\_ROOT/WorldScenery}.
}{}

\IfLanguageName{italian}{
\item \`{e} necessario aggiungere \texttt{/usr/local/share/FlightGear/lib} er la variabile \texttt{LD\_LIBRARY\_PATH}
\item \texttt{FG\_ROOT} ddeve essere impostata con il percorso della directory d'installazione di \FlightGear{}. (Ad esempio : \texttt{/usr/local/share/FlightGear/data}).
\item \texttt{FG\_SCENERY} deve contenere una lista di percorsi di cartelle, separati da '':''. Questa variabile indica a
FlightGear dove cercare gli scenari. (Esempio di variabile su linux: \texttt{\$FG\_ROOT/Scenery:\$FG\_ROOT/WorldScenery}).
}{}

\end{itemize}

\noindent
\ifchinese
在 Bourne shell(及兼容型)里添加这些变量:
\fi
%\IfLanguageName{english}{
%To add these in the Bourne shell (and compatibles):
%}{}
\IfLanguageName{french}{
Pour les ajouter dans le Bourne shell (et compatibles) :
}{}
\IfLanguageName{italian}{
Per aggiungere queste variabili nella Bourne shell (e compatibili):
}{}
\begin{verbatim}
export LD_LIBRARY_PATH=\
    /usr/local/share/FlightGear/lib:$LD_LIBRARY_PATH
export FG_HOME=/usr/local/share/FlightGear
export FG_ROOT=/usr/local/share/FlightGear/data
export FG_SCENERY=$FG_ROOT/Scenery:$FG_ROOT/WorldScenery
\end{verbatim}

\ifchinese
\noindent
或者在 C Shell(其兼容型)
\fi
%\IfLanguageName{english}{
% or in C shell (and compatibles):
%}{}
\IfLanguageName{french}{
 ou en C shell (et compatibles) :
}{}
\IfLanguageName{italian}{
 Per aggiungere queste variabili nella C shell (e compatibili):
}{}

\begin{verbatim}
setenv LD_LIBRARY_PATH=\
  /usr/local/share/FlightGear/lib:$LD_LIBRARY_PATH
setenv FG_HOME=/usr/local/share/FlightGear
setenv FG_ROOT=/usr/local/share/FlightGear/data
setenv FG_SCENERY=\
  $FG_HOME/Scenery:$FG_ROOT/Scenery:$FG_ROOT/WorldScenery
\end{verbatim}

\ifchinese
当你设置好这些环境变量以后,只需要这样简单的启动 \FlightGear{} 
\fi
%\IfLanguageName{english}{
% Once you have these environment variables set up, simply start \FlightGear{} by running
%}{}
\IfLanguageName{french}{
 Une fois que ces variables d'environnement ont \'{e}t\'{e} param\'{e}tr\'{e}es, d\'{e}marrez tout simplement \FlightGear{} en utilisant la commande
}{}
\IfLanguageName{italian}{
 Una volta impostate queste variabili d'ambiente, \`{e} sufficiente avviare \FlightGear{} eseguendo
}{}

\medskip

\texttt{fgfs -$ $-option1 -$ $-option2\dots}
\medskip

\ifchinese
关于详细的命令行选项将会在 ~\ref{options} 小节介绍。
\fi
%\IfLanguageName{english}{
%Command-line options are described in Chapter~\ref{options}.
%}{}
\IfLanguageName{french}{
Les options de ligne de commande sont d\'{e}crites dans le chapitre~\ref{options}.
}{}
\IfLanguageName{italian}{
Le opzioni della riga di comando sono descritte nel capitolo~\ref{options}.
}{}

%%%%%%%%%%%%%%%%%%%%%%%%%%%%%%%%%%%%%%%%%%%%%%%%%%%%%%%%%%%%%%%%%%%%%%%%%%%%%%%%%%%%%%%%%%%%%%%
\ifchinese
\subsection{在 Mac OS X 下启动模拟器}\index{启动 Flightgear!Mac OS X}\index{启动 Flightgear!Mac OS X}
\fi
%\IfLanguageName{english}{
%\section{Launching the simulator under Mac OS X}\index{Launching Flightgear!Mac OS X}\index{Starting Flightgear!Mac OS X}
%}{}
\IfLanguageName{french}{
\section{D\'{e}marrer le simulateur sous Mac OS X}\index{D\'{e}marrer Flightgear!Mac OS X}\index{D\'{e}marrer Flightgear!Mac OS X}
}{}
\IfLanguageName{italian}{
\section{Avvio del simulatore sotto Mac OS X}\index{Avvio del simulatore!Mac OS X}\index{Avvio del simulatore sotto Mac OS X}
}{}
%%%%%%%%%%%%%%%%%%%%%%%%%%%%%%%%%%%%%%%%%%%%%%%%%%%%%%%%%%%%%%%%%%%%%%%%%%%%%%%%%%%%%%%%%%%%%%%

你也可以在 Mac OS X 的命令行里启动模拟器。打开 Terminal.app\\ (在 \texttt{/Applications/Utilities})并输入如下的命令:
\fi
%\IfLanguageName{english}{
%\subsection{Launching from the command line}
%You can also launch the simulator from the command line on Mac OS X. To do so, open Terminal.app (located at \texttt{/Applications/Utilities}) and type the following commands:
%}{}
\IfLanguageName{french}{
\subsection{D\'{e}marrage \`{a} partir de la ligne de commande}
Vous pouvez \'{e}galement d\'{e}marrer le simulateur \`{a} partir de la ligne de commande sur Mac OS X. Pour le faire, ouvrez Terminal.app (situ\'{e} dans \texttt{/Applications/Utilitaires}) et tapez les commandes suivantes :
}{}

\IfLanguageName{italian}{
\subsection{Avvio dalla riga di comando}
Su Mac OS X \`{e} anche possibile avviare il simulatore dalla riga di comando.
Per farlo, aprire Terminal.app (si trova in \texttt{/Applications/Utilities})
e digitare i seguenti comandi:
}{}

\begin{verbatim}
  cd /Applications/FlightGear.app/Contents/Resources
  ./fgfs --option1 --option2 ....
\end{verbatim}

\ifchinese
可以通过查看 \ref{options} 小节获取具体的命令行选项。不像其他平台,若你使用预编译的二进制包,你不需要手动指定 FG\_ROOT 和 FG\_SCENERY 这样的环境变量。
\fi
%\IfLanguageName{english}{
%See chapter \ref{options} for detail information on command line options. Unlike the other platforms, you don't have to manually specify the environment variables such as FG\_ROOT and FG\_SCENERY as long as you use a prebuilt binary package.
%}{}
\IfLanguageName{french}{
Reportez-vous au chapitre \ref{options} pour obtenir plus d'informations d\'{e}taill\'{e}es sur les options en ligne de commande. Contrairement aux autres plate-formes, vous n'avez pas besoin de pr\'{e}ciser manuellement les variables d'environnement comme FG\_ROOT et FG\_SCENERY tant que vous utilisez un paquetage binaire pr\'{e}compil\'{e}.
}{}

\IfLanguageName{italian}{
Vedere il capitolo \ref{options} per informazioni dettagliate sulle opzioni della riga di comando.
A differenza delle altre piattaforme, non \`{e} necessario specificare manualmente le
variabili di ambiente come FG\_ROOT e FG\_SCENERY se si utilizza un pacchetto binario precompilato.
}{}

%%%%%%%%%%%%%%%%%%%%%%%%%%%%%%%%%%%%%%
%%%%% Currently the following section is only available in English and Italian.  So non-italian
%%%%% versions use the English version
%%%%%%%%%%%%%%%%%%%%%%%%%%%%%%%%%%%%%%%%%%%%%%%%%%%%%%%


\IfLanguageName{italian}{
  %%%%%%%%%%%%%%%%%%%%%%%%%%%%%%%%%%%%%%%%%%%%%%%%%%%%%%%%%%%%%%%%%%%%%%%%%%%%%%%%%%%%%%%%%%%%%%%
  \section{Parametri della riga di comando\label{options}}\index{Parametri della riga di comando}
  %%%%%%%%%%%%%%%%%%%%%%%%%%%%%%%%%%%%%%%%%%%%%%%%%%%%%%%%%%%%%%%%%%%%%%%%%%%%%%%%%%%%%%%%%%%%%%%
  Di seguito \`{e} riportato un elenco completo e una breve descrizione delle numerose
  opzioni della riga di comando disponibili per FlightGear. La maggior parte di queste
  opzioni \`{e} impostabile anche attraverso i binari precompilati di FlightGear.
  Se si dispone di opzioni che si utilizzano spesso, \`{e} possibile creare un file di
  preferenze contenente una serie di opzioni della riga di comando che verranno impostate
  automaticamente. \`{e} possibile creare il file con qualsiasi editor di testo
  (per esempio: blocco note, emacs, vi).

  \begin{itemize}
  \item Sui sistemi Unix (incluso Mac OS X), scrivere le opzioni della riga di comando in
  un file con estensione \texttt{.fgfsrc}\index{.fgfsrc} posizionato nella cartella d'installazione
  del simulatore.

  \item In Windows, mettere le opzioni della riga di comando in un file chiamato
  \texttt{system.fgfsrc}
  \index{system.fgfsrc} posizionato nella directory \texttt{FG\_ROOT}
  (ad esempio, \texttt{c:$\backslash$Program Files$\backslash$FlightGear$\backslash$data}).
  \end{itemize}

}

%%%%%%%%%%% CHINESE VERSION %%%%%%%%%
\ifchinese
{
  \section{命令行参数\label{options}}\index{命令行选项}
 下面列出的是 \FlightGear{} 中可用的完整\Index{命令行选项}和其简介。

 如果你想连续多次使用这些配置,可以创建一个包含这些命令行参数的首选项文件,这将会在启动时自动加载这些配置。你可以用任何文本编辑器(记事本、Emacs、vi 任何你喜欢的)来创建这个配置文件。

 \begin{itemize}
  \item 大多数 UNIX 系统(包括 Mac OS X),将这个命令行选项的配置文件命名为 \texttt{.fgfsrc}\index{.fgfsrc},并放到你的 HOME 目录下。

  \item 在 Windows 系统下,将命令行配置文件命名为 \texttt{system.fgfsrc}\index{system.fgfsrc} 然后放到 \texttt{FG\_ROOT} 所在的文件夹下(比如:\texttt{c:$\backslash$Program Files$\backslash$FlightGear$\backslash$data})
  \end{itemize}
}
\fi

%{  %%%%%%%%%  ENGLISH VERSION %%%%%%%

  %%%%%%%%%%%%%%%%%%%%%%%%%%%%%%%%%%%%%%%%%%%%%%%%%%%%%%%%%%%%%%%%%%%%%%%%%%%%%%%%%%%%%%%%%%%%%%%

%  \section{Command line parameters\label{options}}\index{command line options}
  %%%%%%%%%%%%%%%%%%%%%%%%%%%%%%%%%%%%%%%%%%%%%%%%%%%%%%%%%%%%%%%%%%%%%%%%%%%%%%%%%%%%%%%%%%%%%%%
%  Following is a complete list and short description of the numerous \Index{command line options}
%  available for \FlightGear{}. Most of these options are exposed through the \FlightGear{} launcher delivered with the
%  Windows binaries.
%
%  If you have options you re-use continually, you can create a preferences file
%  containing a set of command-line options that will be set automatically. You
%  can create the file with any text editor (notepad, emacs, vi, if you like).
%
%  \begin{itemize}
%  \item On Unix systems (including Mac OS X), put the command line options in a
%  file called \texttt{.fgfsrc}\index{.fgfsrc} in your home directory.
%
%  \item On Windows, put the command line options in a file called \texttt{system.fgfsrc}
%  \index{system.fgfsrc} in the \texttt{FG\_ROOT} directory (e.g.
%  \end{itemize}
%}


\IfLanguageName{italian}{
  %%%%%%%%%%%%%%%%%%%%%%%%%%%%%%%%%%%%%%%%%%%%%%%%%%%%%%%%%%%%%%%%%%%%%%%%%%%%%%%%%%%%%%%%%%%%%%%
  \subsection{Opzioni generali}\index{Opzioni!generali}\label{generaloptions}
  %%%%%%%%%%%%%%%%%%%%%%%%%%%%%%%%%%%%%%%%%%%%%%%%%%%%%%%%%%%%%%%%%%%%%%%%%%%%%%%%%%%%%%%%%%%%%%%
  \begin{itemize}
  \item{\texttt{-$ $-help}}

    Visualizza una breve spiegazione delle pi\`{u} importanti opzioni della riga di comando.

  \item{\texttt{-$ $-help} \texttt{-$ $-verbose}}

    Visualizza una breve spiegazione di tutte le opzioni della riga di comando.

  \item{\texttt{-$ $-version} }

    Visualizza la versione corrente di FlightGear.

  \item{\texttt{-$ $-fg-root={\it PERCORSO}}}

    Indica la cartella d'installazione di FlightGear.

  \item{\texttt{-$ $-fg-scenery={\it PERCORSO}}}
    Consente di specificare il percorso dove FlightGear deve cercare gli
    scenari, nel caso in cui questi non fossero nella posizione di default
    sotto \texttt{\$FG\underline{~}ROOT/Scenery}, questo potrebbe essere
    particolarmente utile nel caso in cui si disponga di scenari installati
    su un CD-ROM.

  \item{\texttt{-$ $-fg-aircraft={\it PERCORSO}}}

    Consente di indicare a FlightGear la cartella contenente gli aeromobili.
    Il valore predefinito \`{e}
    \texttt{\$FG\underline{~}ROOT/Aircraft}.

  \item{\texttt{-$ $-language={\it code}}}

    Imposta la lingua per questa sessione. Ad esempio pl, nl, it, fr, en, de.

  \item{\texttt{-$ $-restore-defaults}}

    Ripristina tutte le impostazioni utente alle impostazioni predefinite.

  \item{\texttt{-$ $-enable-save-on-exit}, \texttt{-$ $-disable-save-on-exit}}

    Attiva o disattiva il salvataggio automatico delle preferenze
    dell'utente all'uscita dal simulatore.

  \item{\texttt{-$ $-enable-freeze}, \texttt{-$ $-disable-freeze}}

    Controlla se FlightGear deve iniziare in pausa o no. Il valore
    predefinito \`{e} ''non in pausa''.

  \item{\texttt{-$ $-enable-auto-coordination}, \texttt{-$ $-disable-auto-coordination}}

    Abilita/disabilita l'\Index{auto-coordinamento} tra gli alettoni e il timone.
    L'Auto-coordinamento \`{e} consigliato per gli utenti senza pedaliera
    o un joystick 'twist'. Il valore predefinito \`{e} ''off''.

  \item{\texttt{-$ $-browser-app={\it PERCORSO}}}

    ndica la posizione del tuo browser web. Per esempio:
    \texttt{-$ $-browser-app=}\\
    \texttt{``C:$\backslash$Program~Files$\backslash$Internet~Explorer$\backslash$iexplore.exe''}
    (Prestare attenzione alle virgolette!).

  \item{\texttt{-$ $-config={\it path}}}

    LCaricare un file contenente propriet\`{a} aggiuntive dato un determinato percorso. Per esempio:
    \texttt{-$ $-config=./Aircraft/X15-set.xml}

  \item{\texttt{-$ $-units-feet}}

    Imposta come unit\`{a} di misura i piedi.

  \item{\texttt{-$ $-units-meters}}

    Imposta come unit\`{a} di misura i metri.


  \end{itemize}
}

\ifchinese
{
%%%%%%%%%%%%%%%%%%%%%%%%%%%%%%%%%%%%%%%%%%%%%%%%%%%%%%%%%%%%%%%%%%%%%%%%%%%%%%%%%%%%%%%%%%%%%%%
\subsection{通用选项}\index{选项!通用}\label{generaloptions}
%%%%%%%%%%%%%%%%%%%%%%%%%%%%%%%%%%%%%%%%%%%%%%%%%%%%%%%%%%%%%%%%%%%%%%%%%%%%%%%%%%%%%%%%%%%%%%%
\begin{itemize}
\item{\texttt{-$ $-launcher}}

  打开启动器(见前文)。
  
\item{\texttt{-$ $-help}}

  显示相关的命令行选项。

\item{\texttt{-$ $-help} \texttt{-$ $-verbose}}

  显示全部命令行选项。

\item{\texttt{-$ $-version} }

  显示当前 \FlightGear{} 的版本。

\item{\texttt{-$ $-fg-root={\it path}}}

  当你没有使用\Index{默认设置}编译时,告诉 \FlightGear{} 去哪里找到程序的根目录。

\item{\texttt{-$ $-fg-scenery={\it path}}}

  允许指定基础地景的路径\index{地景目录!路径},仅当地景没有安装到默认的 \texttt{\$FG\underline{~}ROOT/Scenery} 目录;在你使用 CD-ROM 上的地景时这样做非常有效。

\item{\texttt{-$ $-fg-aircraft={\it path}}}

  用来指定航空器所在的目录,默认是 \texttt{\$FG\underline{~}ROOT/Aircraft}。

\item{\texttt{-$ $-language={\it code}}}

  用来指定会话的语言。比如 pl, nl, it, fr, en, de。

\item{\texttt{-$ $-restore-defaults}}

  重置所有用户设置到默认

\item{\texttt{-$ $-enable-save-on-exit}, \texttt{-$ $-disable-save-on-exit}}

  是否启用模拟器退出时自动保存用户首选项。

\item{\texttt{-$ $-enable-freeze}, \texttt{-$ $-disable-freeze}}

  控制 \FlightGear{} 启动时是否是暂停状态。默认是不暂停。

\item{\texttt{-$ $-enable-auto-coordination}, \texttt{-$ $-disable-auto-coordination}}

  切换是否启用副翼与舵的协调配合。建议没有脚踏板或者游戏杆没有 Z 轴的用户打开自动配合。默认是关闭的。

\item{\texttt{-$ $-browser-app={\it path}}}

  指定你的网络浏览器的位置。例如:\texttt{-$ $-browser-app=}\\
    \texttt{``C:$\backslash$Program~Files$\backslash$Internet~Explorer$\backslash$iexplore.exe''}\\
    (注意:“”之间的空格)

\item{\texttt{-$ $-config={\it path}}}

  载入外部路径的选项。例如:\\
    \texttt{-$ $-config=./Aircraft/X15-set.xml}

\item{\texttt{-$ $-units-feet}}

  使用英制度量衡。

\item{\texttt{-$ $-units-meters}}

  使用米制度量衡。

\end{itemize}
  
}
\fi

% {
%   %%%%%%%%%%%%%%%%%%%%%%%%%%%%%%%%%%%%%%%%%%%%%%%%%%%%%%%%%%%%%%%%%%%%%%%%%%%%%%%%%%%%%%%%%%%%%%%
%   \subsection{General Options}\index{options!general}\label{generaloptions}
%   %%%%%%%%%%%%%%%%%%%%%%%%%%%%%%%%%%%%%%%%%%%%%%%%%%%%%%%%%%%%%%%%%%%%%%%%%%%%%%%%%%%%%%%%%%%%%%%
%   \begin{itemize}
%   \item{\texttt{-$ $-help}}

%     Display the most relevant command line options.

%   \item{\texttt{-$ $-help} \texttt{-$ $-verbose}}

%     Display all command line options.

%   \item{\texttt{-$ $-version} }

%   Display the current \FlightGear{} version.

%   \item{\texttt{-$ $-fg-root={\it path}}}

%     Tells \FlightGear{} where to look for its root data files if you
%     didn't compile it with the \Index{default settings}.

%   \item{\texttt{-$ $-fg-scenery={\it path}}}

%     Allows specification of a path to the base scenery path
%     \index{scenery directory!path}, in case scenery is not at the default
%     position under \texttt{\$FG\underline{~}ROOT/Scenery}; this might
%     be especially useful in case you have scenery on a CD-ROM.

%   \item{\texttt{-$ $-fg-aircraft={\it path}}}

%     Allows specification of a paths to aircraft. Defaults to
%     \texttt{\$FG\underline{~}ROOT/Aircraft}.

%   \item{\texttt{-$ $-language={\it code}}}

%     Select the language for this session. e.g. pl, nl, it, fr, en, de.

%   \item{\texttt{-$ $-restore-defaults}}

%     Reset all user settings to their defaults

%   \item{\texttt{-$ $-enable-save-on-exit}, \texttt{-$ $-disable-save-on-exit}}

%     Enable or disable saving of user-preferences on exit from the simulator.

%   \item{\texttt{-$ $-enable-freeze}, \texttt{-$ $-disable-freeze}}

%     Control whether \FlightGear{} starts paused or not. Defaults to not paused.

%   \item{\texttt{-$ $-enable-auto-coordination}, \texttt{-$ $-disable-auto-coordination}}

%     Switches \Index{auto-co-ordination} between aileron and rudder on/off. Auto-coordination
%     is recommended for users without rudder pedals or a `twist' joystick. Defaults to off.

%   \item{\texttt{-$ $-browser-app={\it path}}}

%     Specify location of your web browser. E.g:
%     \texttt{-$ $-browser-app=}\\
%     \texttt{``C:$\backslash$Program~Files$\backslash$Internet~Explorer$\backslash$iexplore.exe''}
%     (Note the `` '' because of the spaces!).

%   \item{\texttt{-$ $-config={\it path}}}

%     Load additional properties from the given path. E.g:\\
%     \texttt{-$ $-config=./Aircraft/X15-set.xml}

%   \item{\texttt{-$ $-units-feet}}

%     Use feet as the unit of measurement.

%   \item{\texttt{-$ $-units-meters}}

%     Use meters as the unit of measurement.

%   \end{itemize}
% }


\IfLanguageName{italian}{

  %%%%%%%%%%%%%%%%%%%%%%%%%%%%%%%%%%%%%%%%%%%%%%%%%%%%%%%%%%%%%%%%%%%%%%%%%%%%%%%%%%%%%%%%%%%%%%%
  \subsection{Caratteristiche}\index{Opzioni!Caratteristiche}
  %%%%%%%%%%%%%%%%%%%%%%%%%%%%%%%%%%%%%%%%%%%%%%%%%%%%%%%%%%%%%%%%%%%%%%%%%%%%%%%%%%%%%%%%%%%%%%%
  \begin{itemize}

  \item{\texttt{-$ $-enable-ai-models}, \texttt{-$ $-disable-ai-models}}

    Attiva/disattiva la visualizzazione degli altri modelli di aerei nel simulatore.

  \item{\texttt{-$ $-ai-scenario={\it nome}}}

    Imposta uno scenario specifico (ad esempio: \texttt{-$ $-ai-scenario=vinson-demo}).
    Pu\`{o} essere utilizzato pi\`{u} volte.

  \end{itemize}
}

\ifchinese
{
  %%%%%%%%%%%%%%%%%%%%%%%%%%%%%%%%%%%%%%%%%%%%%%%%%%%%%%%%%%%%%%%%%%%%%%%%%%%%%%%%%%%%%%%%%%%%%%%
  \subsection{特性}\index{选项!特性}
  %%%%%%%%%%%%%%%%%%%%%%%%%%%%%%%%%%%%%%%%%%%%%%%%%%%%%%%%%%%%%%%%%%%%%%%%%%%%%%%%%%%%%%%%%%%%%%%
\begin{itemize}

\item{\texttt{-$ $-enable-ai-models}, \texttt{-$ $-disable-ai-models}}
 
  启用或禁用模拟器里的其他航空器/AI 模型。

\item{\texttt{-$ $-ai-scenario={\it name}}}

  启用一个特定的 AI 场景(例如 \texttt{-$ $-ai-scenario=vinson-demo})。可以启用多个。

\end{itemize}
}
\fi
% {
%   %%%%%%%%%%%%%%%%%%%%%%%%%%%%%%%%%%%%%%%%%%%%%%%%%%%%%%%%%%%%%%%%%%%%%%%%%%%%%%%%%%%%%%%%%%%%%%%
%   \subsection{Features}\index{options!features}
%   %%%%%%%%%%%%%%%%%%%%%%%%%%%%%%%%%%%%%%%%%%%%%%%%%%%%%%%%%%%%%%%%%%%%%%%%%%%%%%%%%%%%%%%%%%%%%%%
%   \begin{itemize}

%   \item{\texttt{-$ $-enable-ai-models}, \texttt{-$ $-disable-ai-models}}

%     Enable or disable other aircraft/AI-models in the simulator.

%   \item{\texttt{-$ $-ai-scenario={\it name}}}

%     Enable a specific AI scenario (e.g. \texttt{-$ $-ai-scenario=vinson-demo}). May be used multiple times.

%   \end{itemize}
% }

\IfLanguageName{italian}{
  %%%%%%%%%%%%%%%%%%%%%%%%%%%%%%%%%%%%%%%%%%%%%%%%%%%%%%%%%%%%%%%%%%%%%%%%%%%%%%%%%%%%%%%%%%%%%%%
  \subsection{Suono}\index{Opzioni!Suono}
  %%%%%%%%%%%%%%%%%%%%%%%%%%%%%%%%%%%%%%%%%%%%%%%%%%%%%%%%%%%%%%%%%%%%%%%%%%%%%%%%%%%%%%%%%%%%%%%
  \begin{itemize}
  \item{\texttt{-$ $-enable-sound}, \texttt{-$ $-disable-sound}}

    Attiva/Disattiva i suoni del simulatore.

  \item{\texttt{-$ $-show-sound-devices}}

    Mostra i dispositivi audio disponibili.

  \item{\texttt{-$ $-sound-device={\it device}}}

    Imposta il dispositivo audio da utilizzare.

  \item{\texttt{-$ $-enable-intro-music}, \texttt{-$ $-disable-intro-music}}

    Attiva/disattiva la riproduzione di una musichetta quando FlightGear si avvia.

  \end{itemize}

}

\ifchinese
{
  %%%%%%%%%%%%%%%%%%%%%%%%%%%%%%%%%%%%%%%%%%%%%%%%%%%%%%%%%%%%%%%%%%%%%%%%%%%%%%%%%%%%%%%%%%%%%%%
  \subsection{声音}\index{选项!声音}
  %%%%%%%%%%%%%%%%%%%%%%%%%%%%%%%%%%%%%%%%%%%%%%%%%%%%%%%%%%%%%%%%%%%%%%%%%%%%%%%%%%%%%%%%%%%%%%%
  \begin{itemize}
  \item{\texttt{-$ $-enable-sound}, \texttt{-$ $-disable-sound}}
 
   启用或禁用声音。

  \item{\texttt{-$ $-show-sound-devices}}

   显示可用的声音设备。
  
  \item{\texttt{-$ $-sound-device={\it device}}}
   
   指定音频设备。

  \end{itemize}
}
\fi

% {
%   %%%%%%%%%%%%%%%%%%%%%%%%%%%%%%%%%%%%%%%%%%%%%%%%%%%%%%%%%%%%%%%%%%%%%%%%%%%%%%%%%%%%%%%%%%%%%%%
%   \subsection{Sound}\index{options!sound}
%   %%%%%%%%%%%%%%%%%%%%%%%%%%%%%%%%%%%%%%%%%%%%%%%%%%%%%%%%%%%%%%%%%%%%%%%%%%%%%%%%%%%%%%%%%%%%%%%
%   \begin{itemize}
%   \item{\texttt{-$ $-enable-sound}, \texttt{-$ $-disable-sound}}

%     Enable or disable sound.

%   \item{\texttt{-$ $-show-sound-devices}}

%     Show the available sound devices.

%   \item{\texttt{-$ $-sound-device={\it device}}}

%     Specify the sound device to use for audio.

%   \item{\texttt{-$ $-enable-intro-music}, \texttt{-$ $-disable-intro-music}}

%     Enable or disable playing an audio sample when \FlightGear{} starts up.

%   \end{itemize}
% }


\IfLanguageName{italian}{
  %%%%%%%%%%%%%%%%%%%%%%%%%%%%%%%%%%%%%%%%%%%%%%%%%%%%%%%%%%%%%%%%%%%%%%%%%%%%%%%%%%%%%%%%%%%%%%%
  \subsection{Aerei\index{Aerei}}\index{Opzioni!Aerei}
  %%%%%%%%%%%%%%%%%%%%%%%%%%%%%%%%%%%%%%%%%%%%%%%%%%%%%%%%%%%%%%%%%%%%%%%%%%%%%%%%%%%%%%%%%%%%%%%

  \begin{itemize}
  \item{\texttt{-$ $-aircraft={\it aeromobile}}}

  Carica il velivolo specificato, ad esempio:: \texttt{-$ $-aircraft=c172p}. Per le scelte disponibili
  controllare la directory \texttt{\$FG\underline{~}ROOT/Aircraft}, e cercare i file che terminano in \texttt{``-set.xml''}.
  Quando si specifica il velivolo, eliminare il  \texttt{``-set.xml''} dal nome del file. In alternativa,
  utilizzare l'opzione \texttt{-$ $-show-aircraft} descritta di seguito per elencare i velivoli disponibili.
  Per informazioni sul download dei velivoli, vedere la sezione \ref{install_aircraft}.

  \item{\texttt{-$ $-show-aircraft}}

  Visualizza un elenco ordinato dei tipi di aeromobili attualmente disponibili.

  \item{\texttt{-$ $-min-status={\it status}}}

  Visualizza solo gli aeromobili con determinato stato(per esempio:
  \texttt{alpha}, \texttt{beta}, \texttt{early-production}, \texttt{production}.
  Da utilizzare con \texttt{-$ $-show-aircraft}.

  \item{\texttt{-$ $-aircraft-dir={\it PATH}}}

  Indica il percorso dove FlightGear deve cercare gli aerei.
  Il valore predefinito \`{e} \texttt{\$FG\underline{~}ROOT/Aircraft}.

  \item{\texttt{-$ $-vehicle={\it name of aircraft definition file}}}

  Ha la stessa funzione di \texttt{-$ $-aircraft}.

  \item{\texttt{-$ $-livery={\it Name}}}

  Imposta l'aspetto dell'aeromobile (colorazione).

  \end{itemize}

}

\ifchinese
{
  %%%%%%%%%%%%%%%%%%%%%%%%%%%%%%%%%%%%%%%%%%%%%%%%%%%%%%%%%%%%%%%%%%%%%%%%%%%%%%%%%%%%%%%%%%%%%%%
  \subsection{航空器\index{航空器!选择!飞行器}}\index{选项!航空器!飞行器}
  %%%%%%%%%%%%%%%%%%%%%%%%%%%%%%%%%%%%%%%%%%%%%%%%%%%%%%%%%%%%%%%%%%%%%%%%%%%%%%%%%%%%%%%%%%%%%%%
\begin{itemize}
\item{\texttt{-$ $-aircraft={\it 航空器}}}
   
   载入特定的航空器。比如: \texttt{-$ $-aircraft=c172p}。可以查看 \texttt{\$FG\underline{~}ROOT/Aircraft} 目录来获取可用的航空器,找到文件名结尾是 \texttt{``-set.xml''} 的文件。指定航空器时,去掉文件名后面的 \texttt{``-set.xml''} 即可。另外,用 \texttt{-$ $-show-aircraft} 选项可以打印可用的航空器的列表。关于如何下载和安装航空器,请前往 \ref{install_aircraft} 节。

\item{\texttt{-$ $-show-aircraft}}

  打印一个可用航空器的列表。

\item{\texttt{-$ $-min-status={\it 状态}}}

  按照声明的开发状态来显示航空器列表,可选的值有 \texttt{alpha}, \texttt{beta}, \texttt{early-production}, \texttt{production}。需要配合 \texttt{-$ $-show-aircraft} 选项。

\item{\texttt{-$ $-aircraft-dir={\it PATH}}}

  相对于执行位置的航空器文件所在路径。默认 \texttt{\$FG\underline{~}ROOT/Aircraft}。

\item{\texttt{-$ $-vehicle={\it 航空器定义文件}}}

  与 \texttt{-$ $-aircraft} 相同。

  \item{\texttt{-$ $-livery={\it Name}}}

  设置航空器涂装。

  \end{itemize}
}
\fi

% {
%   %%%%%%%%%%%%%%%%%%%%%%%%%%%%%%%%%%%%%%%%%%%%%%%%%%%%%%%%%%%%%%%%%%%%%%%%%%%%%%%%%%%%%%%%%%%%%%%
%   \subsection{Aircraft\index{aircraft!selection}}\index{options!aircraft}
%   %%%%%%%%%%%%%%%%%%%%%%%%%%%%%%%%%%%%%%%%%%%%%%%%%%%%%%%%%%%%%%%%%%%%%%%%%%%%%%%%%%%%%%%%%%%%%%%

%   \begin{itemize}
%   \item{\texttt{-$ $-aircraft={\it aircraft}}}

%   Load the specified aircraft, for example: \texttt{-$ $-aircraft=c172p}. For available choices
%   check the directory \texttt{\$FG\underline{~}ROOT/Aircraft}, and look for files ending in \texttt{``-set.xml''}.
%   When specifying the aircraft, drop the \texttt{``-set.xml''} from the filename. Alternatively, use
%   the \texttt{-$ $-show-aircraft} option described below to list the available aircraft. For information
%   on downloading additional aircraft, see Section \ref{install_aircraft}.

%   \item{\texttt{-$ $-show-aircraft}}

%   Print a sorted list of the currently available aircraft types.

%   \item{\texttt{-$ $-min-status={\it status}}}

%   Display only those aircraft with a specified minimum declared status, one of
%   \texttt{alpha}, \texttt{beta}, \texttt{early-production}, \texttt{production}. For use with \texttt{-$ $-show-aircraft}.

%   \item{\texttt{-$ $-aircraft-dir={\it PATH}}}

%   Aircraft directory relative to the executable location. Defaults to \texttt{\$FG\underline{~}ROOT/Aircraft}.

%   \item{\texttt{-$ $-vehicle={\it name of aircraft definition file}}}

%   Synonym for \texttt{-$ $-aircraft}.

%   \item{\texttt{-$ $-livery={\it Name}}}

%   Set the aircraft livery.

%   \end{itemize}
% }

\IfLanguageName{italian}{
  %%%%%%%%%%%%%%%%%%%%%%%%%%%%%%%%%%%%%%%%%%%%%%%%%%%%%%%%%%%%%%%%%%%%%%%%%%%%%%%%%%%%%%%%%%%%%%%
  \subsection{Modello di volol}\index{Modello di volo}\index{Opzioni!Modello di volo}\label{flight dynamics model}
  %%%%%%%%%%%%%%%%%%%%%%%%%%%%%%%%%%%%%%%%%%%%%%%%%%%%%%%%%%%%%%%%%%%%%%%%%%%%%%%%%%%%%%%%%%%%%%%
  \begin{itemize}
  \item{\texttt{-$ $-fdm={\it abcd}}}

  Seleziona il modello di base di volo \index{modello di volo}. Le opzioni disponibili sono:
   \texttt{jsb}, \texttt{larcsim}, \texttt{yasim},
  \texttt{magic}, \texttt{balloon}, \texttt{external}, \texttt{pipe}, \texttt{ada}, \texttt{null}.
  TQuesta opzione pu\`{o} essere normalmente ignorata, dato che l'opzione \texttt{-$ $-aircraft}
  impostare automaticamente il FDM.

  \item{\texttt{-$ $-aero={\it aircraft}}}

  Specifica il modello aeronautico dell'aereo da caricare. Questa opzione pu\`{o} essere
  normalmente ignorata, dato che l'opzione \texttt{-$ $-aircraft} dovrebbe impostare
  automaticamente il modello dell'aeromobile.

  \item{\texttt{-$ $-model-hz={\it n}}}

  Esegue l'FDM con il tasso indicato (iterazioni al secondo).

  \item{\texttt{-$ $-speed={\it n}}}

  Imposta la velocit\`{a} dell'FDM rispetto al tempo reale.

  \item{\texttt{-$ $-trim}, \texttt{-$ $-notrim}}

  Trim (o meno) durante l'inizializzazione di JSBSim. L'opzione predefinita \`{e} ``trim''.

  \item{\texttt{-$ $-on-ground}, \texttt{-$ $-in-air}}

  Specifica se partire al livello del suolo (default), o in aria. Se si specifica \texttt{-$ $-in-air}
  necessario impostare anche una quota iniziale utilizzando \texttt{-$ $-altitude}
  , e si consiglia inoltre di impostare una velocit\`{a} iniziale con
  \texttt{-$ $-vc}. Si noti che alcuni aerei (in particolare l'X15) devono essere avviati a mezz'aria.

  \item{\texttt{-$ $-enable-fuel-freeze}, \texttt{-$ $-disable-fuel-freeze}}

  Controlla se il combustibile debba essere consumato normalmente (default) o
  debba essere in quantit\`{a} costante (congelato).

  \end{itemize}
}

\ifchinese
{
  %%%%%%%%%%%%%%%%%%%%%%%%%%%%%%%%%%%%%%%%%%%%%%%%%%%%%%%%%%%%%%%%%%%%%%%%%%%%%%%%%%%%%%%%%%%%%%%
  \subsection{飞行模型}\index{flight dynamics model 飞行动态模型}\index{选项!飞行模型}\label{flight dynamics model}
  %%%%%%%%%%%%%%%%%%%%%%%%%%%%%%%%%%%%%%%%%%%%%%%%%%%%%%%%%%%%%%%%%%%%%%%%%%%%%%%%%%%%%%%%%%%%%%%
  \begin{itemize}
  \item{\texttt{-$ $-fdm={\it abcd}}}

  选择核心的\Index{飞行模型}。可用的选项 \texttt{jsb}, \texttt{larcsim}, \texttt{yasim}, \texttt{magic}, \texttt{balloon}, \texttt{external}, \texttt{pipe}, \texttt{ada}, \texttt{null}。现在这个选项一般可以忽略,\texttt{-$ $-aircraft} 选项已经可以很好的设定 FDM。

\item{\texttt{-$ $-aero={\it aircraft}}}

  指定加载的\Index{航空器空气动力学模型}。现在这个选项一般可以忽略,\texttt{-$ $-aircraft} 选项已经可以很好的设定航空器模型。

 \item{\texttt{-$ $-model-hz={\it n}}}
  
  以此速率(每秒迭代数)运行飞行动态模型。

 \item{\texttt{-$ $-speed={\it n}}}

  让飞行动态模型运行速度快于真实时间。

\item{\texttt{-$ $-trim}, \texttt{-$ $-notrim}}

 当初始化 JSBSim 时是否修正,默认会启用修正。

\item{\texttt{-$ $-on-ground}, \texttt{-$ $-in-air}}

 指定启动时在地面(默认),或者在天空中。如果指定 \texttt{-$ $-in-air} 你必须同时使用 \texttt{-$ $-altitude} 来设定初始高度,你可以用 \texttt{-$ $-vc} 设定初始速度。请注意有些航空器(特别是 X15)必须默认从空中启动。

  \item{\texttt{-$ $-enable-fuel-freeze}, \texttt{-$ $-disable-fuel-freeze}}
 
 控制燃油模式是否恒定(比如 冻结)或者正常消耗(默认)。
 
\end{itemize}
}
\fi

% {
%   %%%%%%%%%%%%%%%%%%%%%%%%%%%%%%%%%%%%%%%%%%%%%%%%%%%%%%%%%%%%%%%%%%%%%%%%%%%%%%%%%%%%%%%%%%%%%%%
%   \subsection{Flight model}\index{flight dynamics model}\index{options!flight model}\label{flight dynamics model}
%   %%%%%%%%%%%%%%%%%%%%%%%%%%%%%%%%%%%%%%%%%%%%%%%%%%%%%%%%%%%%%%%%%%%%%%%%%%%%%%%%%%%%%%%%%%%%%%%
%   \begin{itemize}
%   \item{\texttt{-$ $-fdm={\it abcd}}}

%   Select the core \Index{flight model}. Options are \texttt{jsb}, \texttt{larcsim}, \texttt{yasim},
%   \texttt{magic}, \texttt{balloon}, \texttt{external}, \texttt{pipe}, \texttt{ada}, \texttt{null}.
%   This option can normally be ignored, as the \texttt{-$ $-aircraft} option will set the FDM correctly.

%   \item{\texttt{-$ $-aero={\it aircraft}}}

%   Specifies the \Index{aircraft aeronautical model} to load. This option can normally be ignored, as
%   the \texttt{-$ $-aircraft} option will set the aircraft model correctly.

%   \item{\texttt{-$ $-model-hz={\it n}}}

%   Run the Flight Dynamics Model with this rate (iterations per second).

%   \item{\texttt{-$ $-speed={\it n}}}

%   Run the Flight Dynamics Model this much faster than real time.

%   \item{\texttt{-$ $-trim}, \texttt{-$ $-notrim}}

%   Trim (or not) when initializing JSBSim. Defaults to trim.

%   \item{\texttt{-$ $-on-ground}, \texttt{-$ $-in-air}}

%   Start up at ground level (default), or in the air. If specifying \texttt{-$ $-in-air} you
%   must also set an initial altitude using \texttt{-$ $-altitude}, and may also want to set
%   an initial velocity with \texttt{-$ $-vc}. Note that some aircraft (notably the X15) must
%   be started in mid-air.

%   \item{\texttt{-$ $-enable-fuel-freeze}, \texttt{-$ $-disable-fuel-freeze}}

%   Control whether fuel state is constant (e.g. frozen) or consumed normally (default).

%   \end{itemize}
% }

\IfLanguageName{italian}{
  %%%%%%%%%%%%%%%%%%%%%%%%%%%%%%%%%%%%%%%%%%%%%%%%%%%%%%%%%%%%%%%%%%%%%%%%%%%%%%%%%%%%%%%%%%%%%%%
  \subsection{Posizione iniziale e orientamento}
  \index{Posizione iniziale e orientamento}\index{Opzioni!Posizione iniziale e orientamento}\label{aiportid}
  %%%%%%%%%%%%%%%%%%%%%%%%%%%%%%%%%%%%%%%%%%%%%%%%%%%%%%%%%%%%%%%%%%%%%%%%%%%%%%%%%%%%%%%%%%%%%%%

  \begin{itemize}
  \item{\texttt{-$ $-airport={\it ICAO}}}

  Inizia in un determinato aeroporto. L'aeroporto \`{e} specificato dal suo codice ICAO, ad esempio:
  \texttt{-$ $-airport=KJFK} per l'aeroporto JFK di New York. Per trovare un aeroporto degli Stati
  Uniti senza avere il codice ICAO, provate anteponendo un codice di 3 carattere con `K'.

  \item{\texttt{-$ $-parking-id={\it NUMERO ID}}}

  Inizia nel posto di parcheggio indicato. Da usare con il comando -$ $-airport

  \item{\texttt{-$ $-runway={\it SIGLA}}}

  SInizia sulla soglia della pista indicata (ad esempio 28L). Se non viene
  specificata nessuna pista, si inizier\`{a} sulla pista con minor vento.

  \item{\texttt{-$ $-vor={\it ABCD}}, \texttt{-$ $-ndb={\it ABCD}}, \texttt{-$ $-fix={\it ABCD}}}

  Imposta la posizione di partenza rispetto ad un VOR, NDB o FIX. Utile per praticare approcci.

  \item{\texttt{-$ $-carrier={\it NAME}}}

  Inizia su una portaerei. Vedere la sezione \ref{carrier} per i dettagli delle operazioni sulle portaerei.

  \item{\texttt{-$ $-parkpos={\it NAME}}}

  Inizia in una particolare posizione di parcheggio sul supporto della portaerei.
  Deve essere usato assieme al comando \texttt{-$ $-carrier}.
  Il valore predefinito \`{e} la posizione di lancio.

  \item{\texttt{-$ $-offset-distance={\it nm}}, \texttt{-$ $-offset-azimuth={\it gradi}}}

  Inizia a una determinata distanza e direzione da una posizione impostata con \texttt{-$ $-airport},
  \texttt{-$ $-vor}, \texttt{-$ $-ndb}, \texttt{-$ $-fix}, \texttt{-$ $-carrier}.

  \item{\texttt{-$ $-lon={\it gradi}}, \texttt{-$ $-lat={\it gradi}}}

  \index{latitudine}\index{longitudine}\index{longitudine iniziale}\index{latitudine iniziale}
  Inizia in un particolare longitudine e la latitudine espressa in gradi decimali (sud, ovest negativo).

  \item{\texttt{-$ $-altitude={\it feet}}}

  Inizia in quota specifica. Da usare con il comando \texttt{-$ $-in-air}.
  L'altitude pu\`{o} essere specificata in piedi o in metri a seconda
  dell'unit\`{a} di misura selezionata nelle impostazioni. Si consiglia
  inoltre di impostare una velocit\`{a} iniziale con\texttt{-$ $-vc} per evitare lo stallo immediato.

  \item{\texttt{-$ $-heading={\it gradi}}, \texttt{-$ $-roll={\it gradi}}, \texttt{-$ $-pitch={\it gradi}}}

  Imposta l'\Index{orientamento iniziale} dell'aeromobile. Tutti i valori di default sono:
  0 gradi in direzione Nord, in volo rettilineo e livellato.

  Set the initial \Index{orientation} of the aircraft. All values default to 0 - heading North, in straight and level flight.

  \item{\texttt{-$ $-uBody={\it X}}, \texttt{-$ $-vBody={\it Y}}, \texttt{-$ $-wBody={\it Z}}}

  Imposta la velocit\`{a} iniziale lungo gli assi X, Y e Z. La velocit\`{a} \`{e} in piedi al secondo a meno che non
  sia stata selezionata l'unit\`{a} di misura in metri, nel qual caso l'altitudine sar\`{a} espressa in metri al secondo.

  \item{\texttt{-$ $-vNorth={\it N}}, \texttt{-$ $-vEast={\it E}}, \texttt{-$ $-vDown={\it D}}}

  Imposta la velocit\`{a} iniziale lungo i punti cardinali e gli assi verticali.
  La velocit\`{a} \`{e} in piedi/metri al secondo in base a quale unit\`{a} di misura \`{e} stata scelta nelle impostazioni.

  \item{\texttt{-$ $-vc={\it knots}}, \texttt{-$ $-mach={\it num}}}

  Imposta la velocit\`{a} iniziale in nodi o in Mach. Utile da usare assieme al comando --altitude, a meno che non
  si voglia andare in stallo fin dalla partenza.

  \item{\texttt{-$ $-glideslope={\it degrees}}, \texttt{-$ $-roc={\it fpm}}}

  Imposta l'angolo di discesa iniziale in gradi o come rateo di salita espresso in piedi al minuto.
  Pu\`{o} essere positivo o negativo.

  \end{itemize}
}

\ifchinese
{
  %%%%%%%%%%%%%%%%%%%%%%%%%%%%%%%%%%%%%%%%%%%%%%%%%%%%%%%%%%%%%%%%%%%%%%%%%%%%%%%%%%%%%%%%%%%%%%%
  \subsection{初始位置和方位\label{aiportid}}\index{选项!初始位置}\index{选项!方位}
  %%%%%%%%%%%%%%%%%%%%%%%%%%%%%%%%%%%%%%%%%%%%%%%%%%%%%%%%%%%%%%%%%%%%%%%%%%%%%%%%%%%%%%%%%%%%%%%
\begin{itemize}
\item{\texttt{-$ $-airport={\it ABCD}}}

 从一个特定的\Index{机场}启动。机场用 ICAO 四字代码指定,比如 \texttt{-$ $-airport=KJFK} 表示纽约的 JFK 机场\footnote{纽约肯尼迪国际机场——译者注}。美国机场没有 ICAO 代码的,尝试在三字代码前加上字母“K”\footnote{机场三字代码就是 IATA 代码,即“国际航运协会”代码,如纽约肯尼迪机场(KJFK) 的 IATA 代码即为 JFK。另外国外的一些小型通用航空机场的代码也是三位字母和数字组合的,比如 33W。——译者注}。

\item{\texttt{-$ $-parking-id={\it ABCD}}}

从一个特定的停机位开始。

\item{\texttt{-$ $-runway={\it NNN}}}

从指定跑道的起始点开始(比如 28L)。如果没有指定跑道或者停机位 ID,会分配迎风起飞的跑道。

 \item{\texttt{-$ $-vor={\it ABCD}}, \texttt{-$ $-ndb={\it ABCD}}, \texttt{-$ $-fix={\it ABCD}}}

设定起始位置与 VOR、NDB 或 FIX(定位点)的相对位置。可以用来练习进近。

\item{\texttt{-$ $-carrier={\it NAME}}}

从航空母舰上开始。查看 \ref{carrier} 一节获取航母相关的选项。

\item{\texttt{-$ $-parkpos={\it NAME}}}

指定航母上的起始位置。必须使用 \texttt{-$ $-carrier} 选项。默认在弹射起飞位置。

\item{\texttt{-$ $-offset-distance={\it nm}}, \texttt{-$ $-offset-azimuth={\it deg}}}

设定相对于特定 \texttt{-$ $-airport}, \texttt{-$ $-vor}, \texttt{-$ $-ndb}, \texttt{-$ $-fix}, \texttt{-$ $-carrier} 的距离偏移量和机头朝向偏移量。

 \item{\texttt{-$ $-lon={\it degrees}}, \texttt{-$ $-lat={\it degrees}}}

\index{longitude}\index{latitude}\index{Startup longitude}\index{Startup latitude}
  起始于特定的经度和纬度,以十进制角度表示(西经和南纬用负数)。

\item{\texttt{-$ $-altitude={\it feet}}}

 起始于特定的高度。隐含表示了 \texttt{-$ $-in-air}。高度以英尺为单位,除非你设定了 \texttt{-$ $-units-meters},则会使用米制高度。你也可以同时指定起始速度 \texttt{-$ $-vc} 以防启动后立刻失速。

\item{\texttt{-$ $-heading={\it degrees}}, \texttt{-$ $-roll={\it degrees}}, \texttt{-$ $-pitch={\it degrees}}}

设定航空器的起始\Index{方位(orientation)}。所有值默认都是 0 —— 航向正北平飞。

\item{\texttt{-$ $-uBody={\it X}}, \texttt{-$ $-vBody={\it Y}}, \texttt{-$ $-wBody={\it Z}}}

设定飞机在 X、Y 和 Z 轴上的起始速度。速度以英尺每秒来表示,除非你还选择了 \texttt{-$ $-units-meters},在这种情况下将会用米每秒来表示。

\item{\texttt{-$ $-vNorth={\it N}}, \texttt{-$ $-vEast={\it E}}, \texttt{-$ $-vDown={\it D}}}

设定沿着南北、东西和垂直方向的初始速度。速度是以英尺为单位,除非你设定了 \texttt{-$ $-units-meters},在这种情况下将会用米每秒来表示。

\item{\texttt{-$ $-vc={\it knots}}, \texttt{-$ $-mach={\it num}}}

设定起始空速以节或马赫数为单位。用于在 \texttt{-$ $-altitude} 选项时设定速度,除非你想一启动就立刻失速。

\item{\texttt{-$ $-glideslope={\it gradi}}, \texttt{-$ $-roc={\it fpm}}}

设定下滑道的起始角度以度为单位,或者用爬升率的英尺每分钟作单位。可以是正数也可以是负数。

 \end{itemize}

}
\fi

% {
%   %%%%%%%%%%%%%%%%%%%%%%%%%%%%%%%%%%%%%%%%%%%%%%%%%%%%%%%%%%%%%%%%%%%%%%%%%%%%%%%%%%%%%%%%%%%%%%%
%   \subsection{Initial Position and Orientation\label{aiportid}}\index{options!initial position}\index{options!orientation}
%   %%%%%%%%%%%%%%%%%%%%%%%%%%%%%%%%%%%%%%%%%%%%%%%%%%%%%%%%%%%%%%%%%%%%%%%%%%%%%%%%%%%%%%%%%%%%%%%
%   \begin{itemize}
%   \item{\texttt{-$ $-airport={\it ABCD}}}

%   Start at a specific \Index{airport}. The airport is specified by its ICAO code, e.g. \texttt{-$ $-airport=KJFK} for
%   JFK airport in New York. For US airport without an ICAO code, try prefixing the 3 character code
%   with `K'.

%   \item{\texttt{-$ $-parking-id={\it ABCD}}}

%   Start at a specific parking spot on the airport.

%   \item{\texttt{-$ $-runway={\it NNN}}}

%   Start at the threshold of a specific runway (e.g. 28L). If no runway or parking ID is
%   specified, a runway facing into the wind will be chosen for takeoff.

%   \item{\texttt{-$ $-vor={\it ABCD}}, \texttt{-$ $-ndb={\it ABCD}}, \texttt{-$ $-fix={\it ABCD}}}

%   Set the starting position relative to a VOR, NDB or FIX. Useful for practising approaches.

%   \item{\texttt{-$ $-carrier={\it NAME}}}

%   Start on an aircraft carrier. See \ref{carrier} for details of carrier operations.

%   \item{\texttt{-$ $-parkpos={\it NAME}}}

%   Start at a particular parking position on the carrier. Must be used with \texttt{-$ $-carrier}.
%   Defaults to a catapult launch position.

%   \item{\texttt{-$ $-offset-distance={\it nm}}, \texttt{-$ $-offset-azimuth={\it deg}}}

%   Start at a particular distance and heading from a position set using \texttt{-$ $-airport},
%   \texttt{-$ $-vor}, \texttt{-$ $-ndb}, \texttt{-$ $-fix}, \texttt{-$ $-carrier}.

%   \item{\texttt{-$ $-lon={\it degrees}}, \texttt{-$ $-lat={\it degrees}}}

%   \index{longitude}\index{latitude}\index{Startup longitude}\index{Startup latitude}
%   Start at a particular longitude and latitude, in decimal degrees (south, west negative).

%   \item{\texttt{-$ $-altitude={\it feet}}}

%   Start at specific altitude. Implies \texttt{-$ $-in-air}. Altitude is specified in feet unless you
%   also select \texttt{-$ $-units-meters}, in which case altitude is in meters. You may also want to set
%   an initial velocity with \texttt{-$ $-vc} to avoid stalling immediately.

%   \item{\texttt{-$ $-heading={\it degrees}}, \texttt{-$ $-roll={\it degrees}}, \texttt{-$ $-pitch={\it degrees}}}

%   Set the initial \Index{orientation} of the aircraft. All values default to 0 - heading North, in straight and level flight.

%   \item{\texttt{-$ $-uBody={\it X}}, \texttt{-$ $-vBody={\it Y}}, \texttt{-$ $-wBody={\it Z}}}

%   Set the initial speed along the X, Y and Z axes. Speed is in feet per second unless you
%   also select \texttt{-$ $-units-meters}, in which case altitude is in meters per second.

%   \item{\texttt{-$ $-vNorth={\it N}}, \texttt{-$ $-vEast={\it E}}, \texttt{-$ $-vDown={\it D}}}

%   Set the initial speed along the South-North, West-East and vertical axes. Speed is in feet per second unless you
%   also select \texttt{-$ $-units-meters}, in which case altitude is in meters per second.

%   \item{\texttt{-$ $-vc={\it knots}}, \texttt{-$ $-mach={\it num}}}

%   Set the initial airspeed in knots or as a Mach number. Useful if setting \texttt{-$ $-altitude}, unless you want to stall immediately!

%   \item{\texttt{-$ $-glideslope={\it gradi}}, \texttt{-$ $-roc={\it fpm}}}

%   Set the initial glide slope angle in degrees or as a climb rate in feet per minute. May be positive or negative.

%   \end{itemize}
% }

\IfLanguageName{italian}{
  %%%%%%%%%%%%%%%%%%%%%%%%%%%%%%%%%%%%%%%%%%%%%%%%%%%%%%%%%%%%%%%%%%%%%%%%%%%%%%%%%%%%%%%%%%%%%%%
  \subsection{Opzioni d'ambiente}
  \index{Opzioni d'ambiente}\index{Opzioni!ambiente}
  %%%%%%%%%%%%%%%%%%%%%%%%%%%%%%%%%%%%%%%%%%%%%%%%%%%%%%%%%%%%%%%%%%%%%%%%%%%%%%%%%%%%%%%%%%%%%%%
  \begin{itemize}
  \item{\texttt{-$ $-ceiling={\it Altezza in piedi[:spessore in piedi]}}}

  Imposta un banco di nuvole ad una particolare altezza, e con uno spessore opzionale (di default \`{e} 2000ft).

  \item{\texttt{-$ $-enable-real-weather-fetch}, \texttt{-$ $-disable-real-weather-fetch}}

  Controlla se debbano essere scaricate e utilizzate le informazioni del meteo in tempo reale o no.

  \item{\texttt{-$ $-metar={\it STRINGA METAR}}}

  Utilizza una stringa METAR specifica, ad esempio \texttt{-$ $-metar="XXXX 012345Z 00000KT 99SM CLR 19/M01 A2992"}.
  Il METAR pu\`{o} essere specificato nei formati pi\`{u} comuni (Stati Uniti, Europa). Incompatibile con
  \texttt{-$ $-enable-real-weather-fetch}.

  \item{\texttt{-$ $-random-wind}}

  Imposta un vento con direzione e forza casuali.

  \item{\texttt{-$ $-turbulence={\it n}}}

  Imposta una turbolenza da assente (0.0) a forte (1.0).

  \item{\texttt{-$ $-wind={\it DIREZIONE@VELOCIT\`{A}}}}

  Configura il vento di superficie. La direzione deve essere espressa in gradi e la velocit\`{a} in nodi.
  I valori possono essere specificati come un intervallo utilizzando un separatore;
  Esempio: \texttt{-$ $-wind=180:220@10:15}.

  \item{\texttt{-$ $-season={\it PARAMETRO}}}

  mposta la stagione da usare durante il volo. I parametri validi sono: \texttt{summer} (default), \texttt{winter}.

  \item{\texttt{-$ $-visibility={\it meters}}, \texttt{-$ $-visibility-miles={\it miles}}}

  Imposta la visibilit\`{a} che il simulatore deve usare.

  \end{itemize}
}

\ifchinese
{
  %%%%%%%%%%%%%%%%%%%%%%%%%%%%%%%%%%%%%%%%%%%%%%%%%%%%%%%%%%%%%%%%%%%%%%%%%%%%%%%%%%%%%%%%%%%%%%%
  \subsection{环境选项\index{选项!环境}}
  %%%%%%%%%%%%%%%%%%%%%%%%%%%%%%%%%%%%%%%%%%%%%%%%%%%%%%%%%%%%%%%%%%%%%%%%%%%%%%%%%%%%%%%%%%%%%%%
\begin{itemize}
\item{\texttt{-$ $-ceiling={\it FT\underline{~}ASL[:THICKNESS\underline{~}FT]}}}
 
   设定阴云在特定高度,以及可选的厚度(默认 2000 英尺)。

\item{\texttt{-$ $-enable-real-weather-fetch}, \texttt{-$ $-disable-real-weather-fetch}}

   控制是否启用实时获取天气信息。

\item{\texttt{-$ $-metar={\it METAR STRING}}}

  指定特定的 METAR 字符串。比如 \\ \texttt{-$ $-metar="XXXX 012345Z 00000KT 99SM CLR 19/M01 A2992"} \\ METAR 支持大多数格式(美国,欧洲)。不可与 \texttt{-$ $-enable-real-weather-fetch} 合用。

\item{\texttt{-$ $-random-wind}}

  设定随机的风向和强度。

\item{\texttt{-$ $-turbulence={\it n}}}

  设定乱流的强度,从完全静稳(0.0)到非常剧烈(1.0)。

\item{\texttt{-$ $-wind={\it DIR@SPEED}}}

  指定地面风。风向用度表示,速度用节表示。可以在数值中用冒号表示范围;如:\texttt{-$ $-wind=180:220@10:15}。

\item{\texttt{-$ $-season={\it param}}}

  设定模拟的季节。可选的取值有 \texttt{summer}(默认), \texttt{winter}。

\item{\texttt{-$ $-visibility={\it meters}}, \texttt{-$ $-visibility-miles={\it miles}}}

  设定能见度,用米或英尺表示。

  \end{itemize}
}
\fi
% {

%   %%%%%%%%%%%%%%%%%%%%%%%%%%%%%%%%%%%%%%%%%%%%%%%%%%%%%%%%%%%%%%%%%%%%%%%%%%%%%%%%%%%%%%%%%%%%%%%
%   \subsection{Environment Options\index{options!environment}}
%   %%%%%%%%%%%%%%%%%%%%%%%%%%%%%%%%%%%%%%%%%%%%%%%%%%%%%%%%%%%%%%%%%%%%%%%%%%%%%%%%%%%%%%%%%%%%%%%
%   \begin{itemize}
%   \item{\texttt{-$ $-ceiling={\it FT\underline{~}ASL[:THICKNESS\underline{~}FT]}}}

%   Sets an overcase ceiling at a particular height, and with an optional thickness (defaults to 2000ft).

%   \item{\texttt{-$ $-enable-real-weather-fetch}, \texttt{-$ $-disable-real-weather-fetch}}

%   Control whether real-time weather information is downloaded and used.

%   \item{\texttt{-$ $-metar={\it METAR STRING}}}

%   Use a specific METAR string, e.g. \texttt{-$ $-metar="XXXX 012345Z 00000KT 99SM CLR 19/M01 A2992"}. The METAR may
%   be specified in most common formats (US, European). Incompatible with \texttt{-$ $-enable-real-weather-fetch}.

%   \item{\texttt{-$ $-random-wind}}

%   Sets random wind direction and strength.

%   \item{\texttt{-$ $-turbulence={\it n}}}

%   Sets turbulence from completely calm (0.0) to severe (1.0).

%   \item{\texttt{-$ $-wind={\it DIR@SPEED}}}

%   Specify the surface wind. Direction is in degrees, and speed in knots. Values may be specified as a range
%   by using a colon separator; e.g. \texttt{-$ $-wind=180:220@10:15}.

%   \item{\texttt{-$ $-season={\it param}}}

%   Sets the simulated season. Valid parameters are \texttt{summer} (default), \texttt{winter}.

%   \item{\texttt{-$ $-visibility={\it meters}}, \texttt{-$ $-visibility-miles={\it miles}}}

%   Set the visibility in meters or miles.

%   \end{itemize}
% }

\IfLanguageName{italian}{
  %%%%%%%%%%%%%%%%%%%%%%%%%%%%%%%%%%%%%%%%%%%%%%%%%%%%%%%%%%%%%%%%%%%%%%%%%%%%%%%%%%%%%%%%%%%%%%%
  \subsection{Opzioni di rendering}
  \index{Opzioni di rendering}\index{Opzioni!rendering}
  %%%%%%%%%%%%%%%%%%%%%%%%%%%%%%%%%%%%%%%%%%%%%%%%%%%%%%%%%%%%%%%%%%%%%%%%%%%%%%%%%%%%%%%%%%%%%%%
  \begin{itemize}

  \item{\texttt{-$ $-aspect-ratio-multiplier={\it N}}}

  Imposta un moltiplicatore per il rapporto dello schermo.

  \item{\texttt{-$ $-bpp={\it PROFONDIT\`{A}}}}

  Specifica i bit per pixel da utilizzare.

  \item{\texttt{-$ $-enable-clouds}, \texttt{-$ $-disable-clouds}}

  Attiva (default) o disattiva la \Index{visualizzazione delle nuvole}.

  \item{\texttt{-$ $-enable-clouds3d}, \texttt{-$ $-disable-clouds3d}}

  Attiva (default), o disabilita le \Index{nuvole 3D}. Le nuvole tridimensionali sono
  molto belle e realistiche, ma dipender\`{a} dalla vostra scheda grafica,
  perch\`{e} le schede pi\`{u} vecchie o meno potenti non le supportano.

  \item{\texttt{-$ $-enable-distance-attenuation}, \texttt{-$ $-disable-distance-attenuation}}

  Attiva/disattiva l'attenuazione delle luci rendendo le piste pi\`{u} realistiche.

  \item{\texttt{-$ $-enable-enhanced-lighting}, \texttt{-$ $-disable-enhanced-lighting}}

  Attiva/disattiva un effetto che rende le luci pi\`{u} realistiche durante gli approcci.

  \item{\texttt{-$ $-enable-fullscreen}, \texttt{-$ $-disable-fullscreen}}

  Attiva/disattiva (default) la visualizzazione a schermo intero.

  \item{\texttt{-$ $-enable-game-mode}, \texttt{-$ $-disable-game-mode}}

  Attiva/disattiva la \Index{visualizzazione a schermo intero} per le schede grafiche 3Dfx.

  \item{\texttt{-$ $-enable-horizon-effect}, \texttt{-$ $-disable-horizon-effect}}

  Abilita (default)/disabilita la visuale in prospettiva.

  \item{\texttt{-$ $-enable-mouse-pointer}, \texttt{-$ $-disable-mouse-pointer}}

  Attiva/disattiva (default) il puntatore del \Index{mouse in pi\`{u}}. Utile in modalit\`{a} a schermo intero per le vecchie schede basate su Voodoo.

  \item{\texttt{-$ $-enable-panel}, \texttt{-$ $-disable-panel}}

  Attiva (default)/disattiva il \Index{pannello strumenti}.

  \item{\texttt{-$ $-enable-random-buildings}, \texttt{-$ $-disable-random-building}}

  Abilita/disabilita (default) la visualizzazione di \Index{edifici casuali}.
  Se attivata, questa opzione richiede molta memoria.

  \item{\texttt{-$ $-enable-random-objects}, \texttt{-$ $-disable-random-objects}}

  Abilita (default)/disabilita la visualizzazione di \Index{oggetti casuali negli scenari}.

  \item{\texttt{-$ $-enable-random-vegetation}, \texttt{-$ $-disable-random-vegetation}}

  Attiva (default)/disattiva la visualizzazione di \Index{vegetazione casuale} come alberi.
  Richiede una scheda grafica che supporti Shader GLSL (alcune schede grafiche meno
  recenti o meno potenti non lo fanno).

  \item{\texttt{-$ $-enable-rembrandt}, \texttt{-$ $-disable-rembrandt}}

  Abilita/disabilita (impostazione predefinita) una funzione sperimentale che
  include una maggiore illuminazione e ombre in tempo reale.

  \item{\texttt{-$ $-enable-skyblend}, \texttt{-$ $-disable-skyblend}}

  Attiva (default)/disabilita l'appannamento/foschia.

  \item{\texttt{-$ $-enable-specular-highlight}, \texttt{-$ $-disable-specular-highlight}}

  Attiva (default)/disabilita la visualizzazione delle luci speculari.

  \item{\texttt{-$ $-enable-splash-screen}, \texttt{-$ $-disable-splash-screen}}

  Attiva/disattiva (default) il logo rotante della 3DFX quando la scheda acceleratrice viene inizializzata (solo per 3DFX).

  \item{\texttt{-$ $-enable-textures}, \texttt{-$ $-disable-textures}}

  Attiva (default)/disabilita l'uso di texture.

  \item{\texttt{-$ $-enable-wireframe}, \texttt{-$ $-disable-wireframe}}

  Abilita/disabilita (default) la visualizzazione dei wireframe.
  Se volete sapere come sono fatti internamente gli scenari di \FlightGear{},
  provate ad attivare questa opzione! \index{wireframe}

  \item{\texttt{-$ $-fog-disable}, \texttt{-$ $-fog-fastest}, \texttt{-$ $-fog-nicest}}

  mposta il livello di visualizzazione della \Index{nebbia}. Per ridurre gli sforzi del rendering,
  le zone lontane spariscono nella nebbia per impostazione predefinita. Se si disattiva la nebbia
  (\texttt{-$ $-fog-disable}) si vedr\`{a} pi\`{u} lontano, ma diminuiranno i frame rate.
  Utilizzando\texttt{-$ $-fog-fastest} si visualizzer\`{a} una nebbia meno realistica,
  per aumentare il \Index{frame rate}. L'impostazione predefinita \`{e}  \texttt{-$ $-fog-nicest}
  (la nebbia pi\`{u} realistica).

  \item{\texttt{-$ $-fov={\it gradi}}}

  Imposta il \Index{campo visivo} in gradi. L'impostazione predefinita \`{e} 55.0.

  \item{\texttt{-$ $-materials-file={\it file}}}

  Indica i file dati utilizzati per il rendering dei paesaggi.
  La configurazione di default \`{e}:  \texttt{Materials/regions/materials.xml}.

  \item{\texttt{-$ $-geometry={\it WWWxHHH}}}

  Definisce la risoluzione della finestra/schermo.
  Ad esempio \texttt{-$ $-geometry=1024x768}.\index{risoluzione della finestra}.

  \item{\texttt{-$ $-shading-smooth}, \texttt{-$ $-shading-flat}}

  Dice al simulatore se utilizzare l'ombreggiatura uniforme (default), o
  l'ombreggiatura piatta che \`{e} pi\`{u} veloce ma meno bella.

  \item{\texttt{-$ $-texture-filtering={\it N}}}

  Configura il filtro anisotropico della texture. I valori disponibili sono: 1
  (impostazione predefinita), 2, 4, 8 o 16.

  \item{\texttt{-$ $-view-offset={\it xxx}}}

  Consente di direzionare la visualizzazione predefinita in avanti
  rispetto ad un angolo di 0. I valori possibili sono \texttt{LEFT, RIGHT, CENTER}
  , o un numero specifico di gradi. Utile per la visualizzazione multi-finestra.

  \end{itemize}

}

\ifchinese
{
  %%%%%%%%%%%%%%%%%%%%%%%%%%%%%%%%%%%%%%%%%%%%%%%%%%%%%%%%%%%%%%%%%%%%%%%%%%%%%%%%%%%%%%%%%%%%%%%
  \subsection{渲染选项\index{选项!渲染}}
  %%%%%%%%%%%%%%%%%%%%%%%%%%%%%%%%%%%%%%%%%%%%%%%%%%%%%%%%%%%%%%%%%%%%%%%%%%%%%%%%%%%%%%%%%%%%%%%
 \begin{itemize}

  \item{\texttt{-$ $-aspect-ratio-multiplier={\it N}}}

    设定显示器高宽比的倍数。

\item{\texttt{-$ $-bpp={\it depth}}}

   指定每个像素的位数。

 \item{\texttt{-$ $-enable-clouds}, \texttt{-$ $-disable-clouds}}

   启用(默认)或者禁用\Index{云层}。

\item{\texttt{-$ $-enable-clouds3d}, \texttt{-$ $-disable-clouds3d}}

  启用(默认)或者禁用\Index{3D 云层}。非常漂亮,但取决于你的显卡对 GLSL 着色器的支持,一些老显卡或不那么强的显卡可能不支持。

\item{\texttt{-$ $-enable-distance-attenuation}, \\ \texttt{-$ $-disable-distance-attenuation}}

  启用或禁用更加真实的跑道灯和进近灯衰减。

\item{\texttt{-$ $-enable-enhanced-lighting}, \texttt{-$ $-disable-enhanced-lighting}}

  启用或禁用更加真实的跑道灯和进近灯。

 \item{\texttt{-$ $-enable-fullscreen}, \texttt{-$ $-disable-fullscreen}}

  启用,禁用(默认)全屏模式。

\item{\texttt{-$ $-enable-game-mode}, \texttt{-$ $-disable-game-mode}}

  为 3DFX 显卡启用或禁用\Index{全屏显示}。

\item{\texttt{-$ $-enable-horizon-effect}, \texttt{-$ $-disable-horizon-effect}}

  启用或禁用地平线附近天体的拉长效果。

 \item{\texttt{-$ $-enable-mouse-pointer}, \texttt{-$ $-disable-mouse-pointer}}

  启用或禁用(默认)额外的\Index{鼠标指针}。对老旧 Voodoo 显卡在全屏模式时很好用。

\item{\texttt{-$ $-enable-panel}, \texttt{-$ $-disable-panel}}

 启用(默认)或禁用\Index{仪表板}。

\item{\texttt{-$ $-enable-random-buildings}, \texttt{-$ $-disable-random-building}}

 启用或禁用(默认)随机建筑物。注意随机建筑物会消耗更多内存。

\item{\texttt{-$ $-enable-random-objects}, \texttt{-$ $-disable-random-objects}}

 启用(默认)或禁用随机地景物体。

 \item{\texttt{-$ $-enable-random-vegetation}, \texttt{-$ $-disable-random-vegetation}}

 启用(默认)或禁用随机的植物,比如树。需要支持 GLSL 着色器的显卡。在老旧的或者不强大的显卡下不要启用。

\item{\texttt{-$ $-enable-rembrandt}, \texttt{-$ $-disable-rembrandt}}

 启用或禁用(默认)一些专家级特效包括增强的灯光和实时阴影。

\item{\texttt{-$ $-enable-skyblend}, \texttt{-$ $-disable-skyblend}}

 启用(默认)或禁用雾/霾。

\item{\texttt{-$ $-enable-specular-highlight}, \texttt{-$ $-disable-specular-highlight}}

  启用(默认)或禁用高光反射效果。

\item{\texttt{-$ $-enable-splash-screen}, \texttt{-$ $-disable-splash-screen}}

  启用或禁用(默认)加速板初始化时旋转 3DFX 图标(只用于 3DFX)。

\item{\texttt{-$ $-enable-textures}, \texttt{-$ $-disable-textures}}

  启用(默认)或禁用材质纹理。

\item{\texttt{-$ $-enable-wireframe}, \texttt{-$ $-disable-wireframe}}

  启用或禁用(默认)线帧。如果你想知道 \FlightGear{} 的内部运作机制,可以试试这个!\index{wireframe}

\item{\texttt{-$ $-fog-disable}, \texttt{-$ $-fog-fastest}, \texttt{-$ $-fog-nicest}}

  设定雾的级别,默认会将\Index{雾}里的渲染效果切断。如果禁用\Index{雾}将可以看到更远,但是帧速率将会下降。使用 \texttt{-$ $-fog-fastest} 可以体验不那么真实的雾,可以提升\Index{帧速率}。默认是 \texttt{-$ $-fog-nicest}。

\item{\texttt{-$ $-fov={\it degrees}}}

 设定\Index{视野角度}。默认是 55.0 度。

\item{\texttt{-$ $-materials-file={\it file}}}

 指定渲染地景用的材质图文件。默认是:\\\texttt{Materials/regions/materials.xml}

\item{\texttt{-$ $-geometry={\it WWWxHHH}}

  定义窗口/屏幕的分辨率。例如 \texttt{-$ $-geometry=1024x768}。\index{窗口尺寸}

\item{\texttt{-$ $-shading-smooth}, \texttt{-$ $-shading-flat}}

  使用平滑着色(默认),或者使用更快速但不那么漂亮的平面着色。

\item{\texttt{-$ $-texture-filtering={\it N}}}

  配置各向异性纹理过滤。取值有 1 (默认)、2、4、8 或 16。

\item{\texttt{-$ $-view-offset={\it xxx}}}

  可基于默认的正前方视野角度设置偏移角度。可取的值有 \texttt{LEFT, RIGHT, CENTER},或者角度数值。用于多窗口显示时。\index{偏移量}
\end{itemize}
}
\fi

\iffalse
{
  %%%%%%%%%%%%%%%%%%%%%%%%%%%%%%%%%%%%%%%%%%%%%%%%%%%%%%%%%%%%%%%%%%%%%%%%%%%%%%%%%%%%%%%%%%%%%%%
  \subsection{Rendering Options\index{options!rendering}}
  %%%%%%%%%%%%%%%%%%%%%%%%%%%%%%%%%%%%%%%%%%%%%%%%%%%%%%%%%%%%%%%%%%%%%%%%%%%%%%%%%%%%%%%%%%%%%%%
  \begin{itemize}

  \item{\texttt{-$ $-aspect-ratio-multiplier={\it N}}}

  Set a multiplier for the display aspect ratio.

  \item{\texttt{-$ $-bpp={\it depth}}}

  Specify the bits per pixel.

  \item{\texttt{-$ $-enable-clouds}, \texttt{-$ $-disable-clouds}}

  Enable (default) or disable \Index{cloud layers}.

  \item{\texttt{-$ $-enable-clouds3d}, \texttt{-$ $-disable-clouds3d}}

  Enable (default), disable \Index{3D clouds}. Very pretty, but depend on your graphics card supporting
  GLSL Shaders, which some older, or less powerful graphics cards do not.

  \item{\texttt{-$ $-enable-distance-attenuation}, \texttt{-$ $-disable-distance-attenuation}}

  Enable or disable more realistic runway and approach light attenuation.

  \item{\texttt{-$ $-enable-enhanced-lighting}, \texttt{-$ $-disable-enhanced-lighting}}

  Enable or disable more realistic runway and approach lights.

  \item{\texttt{-$ $-enable-fullscreen}, \texttt{-$ $-disable-fullscreen}}

  Enable, disable (default) full screen mode.

  \item{\texttt{-$ $-enable-game-mode}, \texttt{-$ $-disable-game-mode}}

  Enable or disable \Index{full screen display} for 3DFX graphics cards.

  \item{\texttt{-$ $-enable-horizon-effect}, \texttt{-$ $-disable-horizon-effect}}

  Enable (default), disable celestial body growth illusion near the horizon.

  \item{\texttt{-$ $-enable-mouse-pointer}, \texttt{-$ $-disable-mouse-pointer}}

  Enable, disable (default) extra \Index{mouse pointer}. Useful in full screen mode for old Voodoo based cards.

  \item{\texttt{-$ $-enable-panel}, \texttt{-$ $-disable-panel}}

  Enable (default) the \Index{instrument panel}.

  \item{\texttt{-$ $-enable-random-buildings}, \texttt{-$ $-disable-random-building}}

  Enable, disable (default) random buildings. Note that random buildings take up a lot of memory.

  \item{\texttt{-$ $-enable-random-objects}, \texttt{-$ $-disable-random-objects}}

  Enable (default), disable random scenery objects.

  \item{\texttt{-$ $-enable-random-vegetation}, \texttt{-$ $-disable-random-vegetation}}

  Enable (default), disable random vegetation such as trees. Requires a graphics card that
  supports GLSL Shaders, which some older, or less powerful graphics cards do not.

  \item{\texttt{-$ $-enable-rembrandt}, \texttt{-$ $-disable-rembrandt}}

  Enable, disable (default) an experimental feature that includes
   enhanced lighting and realtime shadows.

  \item{\texttt{-$ $-enable-skyblend}, \texttt{-$ $-disable-skyblend}}

  Enable (default), disable fogging/haze.

  \item{\texttt{-$ $-enable-specular-highlight}, \texttt{-$ $-disable-specular-highlight}}

  Enable (default), disable specular highlights.

  \item{\texttt{-$ $-enable-splash-screen}, \texttt{-$ $-disable-splash-screen}}

  Enable or disable (default) the rotating 3DFX logo when the accelerator board gets initialized (3DFX only).

  \item{\texttt{-$ $-enable-textures}, \texttt{-$ $-disable-textures}}

  Enable (default), disable use of textures.

  \item{\texttt{-$ $-enable-wireframe}, \texttt{-$ $-disable-wireframe}}

  Enable, disable (default) display of wireframes. If you want to know what the world of
  \FlightGear{} looks like internally, try this!\index{wireframe}

  \item{\texttt{-$ $-fog-disable}, \texttt{-$ $-fog-fastest}, \texttt{-$ $-fog-nicest}}

  Set the fog level. To cut down the rendering efforts, distant regions vanish in \Index{fog} by default.
  If you disable \Index{fog} you'll see farther, but your frame rates will drop. Using \texttt{-$ $-fog-fastest}
  will display a less realistic fog, by increase \Index{frame rate}. Default is \texttt{-$ $-fog-nicest}.

  \item{\texttt{-$ $-fov={\it degrees}}}

  Sets the \Index{field of view} in degrees. Default is 55.0.

  \item{\texttt{-$ $-materials-file={\it file}}}

  Specify the materials file used to render the scenery.
  Default: \texttt{Materials/regions/materials.xml}.

  \item{\texttt{-$ $-geometry={\it WWWxHHH}}}

  Defines the window/screen resolution. E.g. \texttt{-$ $-geometry=1024x768}.\index{window size}.

  \item{\texttt{-$ $-shading-smooth}, \texttt{-$ $-shading-flat}}

  Use smooth shading (default), or flat shading which is faster but less pretty.

  \item{\texttt{-$ $-texture-filtering={\it N}}}

  Configure anisotropic texture filtering. Values are 1 (default), 2, 4, 8 or 16.

  \item{\texttt{-$ $-view-offset={\it xxx}}}

  Allows setting the default forward view direction as an offset from straight ahead.
  Possible values are \texttt{LEFT, RIGHT, CENTER}, or a specific number of degrees.
  Useful for multi-window display.\index{offset}
  \end{itemize}
}
\fi 
\IfLanguageName{italian}{
  %%%%%%%%%%%%%%%%%%%%%%%%%%%%%%%%%%%%%%%%%%%%%%%%%%%%%%%%%%%%%%%%%%%%%%%%%%%%%%%%%%%%%%%%%%%%%%%
  \subsection{Opzioni HUD\index{HUD}\index{Opzioni!HUD}}
  %%%%%%%%%%%%%%%%%%%%%%%%%%%%%%%%%%%%%%%%%%%%%%%%%%%%%%%%%%%%%%%%%%%%%%%%%%%%%%%%%%%%%%%%%%%%%%%
  \begin{itemize}
  \item{\texttt{-$ $-enable-anti-alias-hud}, \texttt{-$ $-disable-anti-alias-hud}}

  Controlla se l'\Index{HUD} (Head Up Display) debba venire avviato in modalit\`{a} anti-alias.

  \item{\texttt{-$ $-enable-hud}, \texttt{-$ $-disable-hud}}

  Controlla se debba venire visualizzato l'HUD o meno.
  Il valore predefinito \`{e} \texttt{-$ $-disable-hud}

  \item{\texttt{-$ $-enable-hud-3d}, \texttt{-$ $-disable-hud-3d}}

  Controlla se debba venire visualizzato l'HUD o meno.
  Il valore predefinito \`{e} \texttt{-$ $-disable-hud-3d}

  \item{\texttt{-$ $-hud-culled}, \texttt{-$ $-hud-tris}}

  Visualizza la percentuale di triangoli utilizzati nell'HUD.
  Principalmente di interesse per gli sviluppatori di grafica.

  \end{itemize}
}
\ifchinese
{
 %%%%%%%%%%%%%%%%%%%%%%%%%%%%%%%%%%%%%%%%%%%%%%%%%%%%%%%%%%%%%%%%%%%%%%%%%%%%%%%%%%%%%%%%%%%%%%%
  \subsection{HUD 选项\index{HUD}\index{选项!HUD}}
 %%%%%%%%%%%%%%%%%%%%%%%%%%%%%%%%%%%%%%%%%%%%%%%%%%%%%%%%%%%%%%%%%%%%%%%%%%%%%%%%%%%%%%%%%%%%%%%
\begin{itemize}
  \item{\texttt{-$ $-enable-anti-alias-hud}, \texttt{-$ $-disable-anti-alias-hud}}

  控制是否对\Index{HUD}(\textbf{H}ead \textbf{U}p  \textbf{D}isplay,平视显示)启用防锯齿。

\item{\texttt{-$ $-enable-hud}, \texttt{-$ $-disable-hud}}

  控制是否显示 HUD,默认为禁用。

\item{\texttt{-$ $-enable-hud-3d}, \texttt{-$ $-disable-hud-3d}}

  控制是否显示 3D HUD。默认为禁用。

 \item{\texttt{-$ $-hud-culled}, \texttt{-$ $-hud-tris}}

  显示三角形剔除的百分比,或者 HUD 中三角形渲染的数量。主要方便图形开发者。 

\end{itemize}
}
\fi
\iffalse
{
  %%%%%%%%%%%%%%%%%%%%%%%%%%%%%%%%%%%%%%%%%%%%%%%%%%%%%%%%%%%%%%%%%%%%%%%%%%%%%%%%%%%%%%%%%%%%%%%
  \subsection{HUD Options\index{HUD}\index{options!HUD}}
  %%%%%%%%%%%%%%%%%%%%%%%%%%%%%%%%%%%%%%%%%%%%%%%%%%%%%%%%%%%%%%%%%%%%%%%%%%%%%%%%%%%%%%%%%%%%%%%
  \begin{itemize}
  \item{\texttt{-$ $-enable-anti-alias-hud}, \texttt{-$ $-disable-anti-alias-hud}}

  Control whether the \Index{HUD} (\textbf{H}ead \textbf{U}p  \textbf{D}isplay) is shown anti-aliased.

  \item{\texttt{-$ $-enable-hud}, \texttt{-$ $-disable-hud}}

  Control whether the HUD is displayed. Defaults to disabled.

  \item{\texttt{-$ $-enable-hud-3d}, \texttt{-$ $-disable-hud-3d}}

  Control whether the 3D HUD is displayed. Defaults to disabled.

  \item{\texttt{-$ $-hud-culled}, \texttt{-$ $-hud-tris}}

  Display the percentage of triangles culled, or the number of triangles rendered in the HUD. Mainly
  of interest to graphics developers.

  \end{itemize}
}
\fi

\IfLanguageName{italian}{
  %%%%%%%%%%%%%%%%%%%%%%%%%%%%%%%%%%%%%%%%%%%%%%%%%%%%%%%%%%%%%%%%%%%%%%%%%%%%%%%%%%%%%%%%%%%%%%%
  \subsection{Opzioni di sistema degli aeromobili}
  \index{Opzioni di sistema degli aeromobili}\index{Opzioni!sistema degli aeromobili}
  %%%%%%%%%%%%%%%%%%%%%%%%%%%%%%%%%%%%%%%%%%%%%%%%%%%%%%%%%%%%%%%%%%%%%%%%%%%%%%%%%%%%%%%%%%%%%%%

  \begin{itemize}
  \item{\texttt{-$ $-adf={\it [radiale:]frequenza}}}

  Imposta la frequenza ADF e la radiale.

  \item{\texttt{-$ $-com1={\it frequenza}}, \texttt{-$ $-com2={\it frequenza}}}

  Imposta la frequenza radio COM1/COM2.

  \item{\texttt{-$ $-dme={\it {nav1|nav2|frequenza}}}}

  Imposta il DME per NAV1, NAV2, o per una frequenza/radiale specifica.

  \item{\texttt{-$ $-failure={\it sistema}}}

  Guasta un sistema specifico dell'aeromobile. I sistemi validi sono: \texttt{pitot}, \texttt{static},
  \texttt{vacuum}, \texttt{electrical}. Usare pi\`{u} volte il comando per guastare pi\`{u} sistemi.

  \item{\texttt{-$ $-nav1={\it [radiale:]frequenza}}, \texttt{-$ $-nav2={\it [radiale:]frequenza}}}

  Imposta una determinata frequenza radio/radiale per il NAV1/NAV2.

  \end{itemize}
}
\ifchinese
{
%%%%%%%%%%%%%%%%%%%%%%%%%%%%%%%%%%%%%%%%%%%%%%%%%%%%%%%%%%%%%%%%%%%%%%%%%%%%%%%%%%%%%%%%%%%%%%%
  \subsection{航空器系统选项\index{Avionics 航空电子}\index{选项!航空器系统}}
  %%%%%%%%%%%%%%%%%%%%%%%%%%%%%%%%%%%%%%%%%%%%%%%%%%%%%%%%%%%%%%%%%%%%%%%%%%%%%%%%%%%%%%%%%%%%%%%
\begin{itemize}
  \item{\texttt{-$ $-adf={\it [radial:]frequency}}}
 
  设置 ADF 频率和径向。

\item{\texttt{-$ $-com1={\it frequency}}, \texttt{-$ $-com2={\it frequency}}}

 设置 COM1/COM2 无线电频率。

\item{\texttt{-$ $-dme={\it {nav1|nav2|frequency}}}}

 设置 DME 以 NAV1、NAV2 或指定特定的频率和径向。

\item{\texttt{-$ $-failure={\it system}}}

 设定航空器的失效系统。可用的系统有  \texttt{pitot}, \texttt{static}, \texttt{vacuum}, \texttt{electrical}。使用多次可以设置多个失效故障。

\item{\texttt{-$ $-nav1={\it [radial:]frequency}}, \texttt{-$ $-nav2={\it [radial:]frequency}}}

 设定 NAV1/NAV2 的无线电频率和径向。

\end{itemize}
}
\fi
\iffalse
{
  %%%%%%%%%%%%%%%%%%%%%%%%%%%%%%%%%%%%%%%%%%%%%%%%%%%%%%%%%%%%%%%%%%%%%%%%%%%%%%%%%%%%%%%%%%%%%%%
  \subsection{Aircraft System Options\index{Avionics}\index{options!Aircraft Systems}}
  %%%%%%%%%%%%%%%%%%%%%%%%%%%%%%%%%%%%%%%%%%%%%%%%%%%%%%%%%%%%%%%%%%%%%%%%%%%%%%%%%%%%%%%%%%%%%%%
  \begin{itemize}
  \item{\texttt{-$ $-adf={\it [radial:]frequency}}}

  Set the ADF frequency and radial.

  \item{\texttt{-$ $-com1={\it frequency}}, \texttt{-$ $-com2={\it frequency}}}

  Set the COM1/COM2 radio frequency.

  \item{\texttt{-$ $-dme={\it {nav1|nav2|frequency}}}}

  Set the DME to NAV1, NAV2, or a specific frequency and radial.

  \item{\texttt{-$ $-failure={\it system}}}

  Fail a specific aircraft system. Valid systems are \texttt{pitot}, \texttt{static},
  \texttt{vacuum}, \texttt{electrical}. Specify multiple times to fail multiple systems.

  \item{\texttt{-$ $-nav1={\it [radial:]frequency}}, \texttt{-$ $-nav2={\it [radial:]frequency}}}

  Set the NAV1/NAV2 radio frequency and radial.

  \end{itemize}
}
\fi

\IfLanguageName{italian}{
  %%%%%%%%%%%%%%%%%%%%%%%%%%%%%%%%%%%%%%%%%%%%%%%%%%%%%%%%%%%%%%%%%%%%%%%%%%%%%%%%%%%%%%%%%%%%%%%
  \subsection{Opzioni di tempo}
  \index{Opzioni di tempo}\index{Opzioni!tempo}
  %%%%%%%%%%%%%%%%%%%%%%%%%%%%%%%%%%%%%%%%%%%%%%%%%%%%%%%%%%%%%%%%%%%%%%%%%%%%%%%%%%%%%%%%%%%%%%%
  \begin{itemize}

  \item{\texttt{-$ $-enable-clock-freeze}, \texttt{-$ $-disable-clock-freeze}}

  Controlla se il tempo debba avanzare normalmente o essere fermo (congelato).

  \item{\texttt{-$ $-start-date-gmt={\it yyyy:mm:dd:hh:mm:ss}},
  \texttt{-$ $-start-date-lat={\it yyyy:mm:dd:hh:mm:ss}},
  \texttt{-$ $-start-date-sys={\it yyyy:mm:dd:hh:mm:ss}}}

  Indicano l'esatta ora di partenza/data. Le tre funzioni differiscono tra loro
  perch\`{e} la prima prende come riferimento il Greenwich Mean Time, la seconda
  usa l'ora locale del volo virtuale, e la terza l'ora locale del computer.
  Incompatibile con \texttt{-$ $-time-match-local}, \texttt{-$ $-time-match-real}.

  \item{\texttt{-$ $-time-match-local}, \texttt{-$ $-time-match-real}}

  Quando si utilizza l'opzione {-$ $-time-match-real}(default), la data del
  simulatore viene letta dal clock del sistema, e viene utilizzata cos\`{i}
  com'\`{e}. Quando il volo virtuale \`{e} ambientato nello stesso fuso orario
  in cui si trova il computer, questo pu\`{o} essere affidabile, perch\'{e} gli
  orologi sono sincronizzati. Tuttavia, quando si vola in una parte diversa del
  mondo, potrebbe non essere il caso, perch\'{e} c'\`{e} un numero di ore di
  differenza tra la posizione del computer e la posizione del proprio volo virtuale.

  L'opzione {-$ $-time-match-local} tsi occupa di questo calcolando la differenza
  del fuso orario tra il reale fuso orario del mondo e la posizione del volo virtuale,
  rendendo cos\`{i} i due orari sincronizzati.

  Incompatibile con \texttt{-$ $-start-date-gmt}, \texttt{-$ $-start-date-lat}, \texttt{-$ $-start-date-sys}.

  \item{\texttt{-$ $-time-offset={\it [+-]hh:mm:ss}}}

  Questo comando consente di specificare manualmente la differenza tra
  l'orario del proprio computer e quello desiderato per il volo virtuale.

  \item{\texttt{-$ $-timeofday={\it param}}}

  Imposta l'ora del giorno. I parametri validi sono: \texttt{real} (ora reale),
  \texttt{dawn} (alba), \texttt{morning} (mattina),
  \texttt{noon} (mezzogiorno), \texttt{afternoon} (pomeriggio),
  \texttt{dusk} (tramonto), \texttt{evening} (sera) e \texttt{midnight} (mezzanotte).

  \end{itemize}
}
\ifchinese
{
 %%%%%%%%%%%%%%%%%%%%%%%%%%%%%%%%%%%%%%%%%%%%%%%%%%%%%%%%%%%%%%%%%%%%%%%%%%%%%%%%%%%%%%%%%%%%%%%
  \subsection{时间选项}\index{时间选项}\index{选项!时间}
  %%%%%%%%%%%%%%%%%%%%%%%%%%%%%%%%%%%%%%%%%%%%%%%%%%%%%%%%%%%%%%%%%%%%%%%%%%%%%%%%%%%%%%%%%%%%%%%
\begin{itemize}

  \item{\texttt{-$ $-enable-clock-freeze}, \texttt{-$ $-disable-clock-freeze}}

   控制时间推移是正常或者冻结。
 
\item{\texttt{-$ $-start-date-gmt={\it yyyy:mm:dd:hh:mm:ss}},\\
  \texttt{-$ $-start-date-lat={\it yyyy:mm:dd:hh:mm:ss}},\\
  \texttt{-$ $-start-date-sys={\it yyyy:mm:dd:hh:mm:ss}}}

  指定精确的起始时间/日期。这三种功能的区别是指定不同的参考点,分别是格林威治时间,虚拟飞行时的本地时间,或者你的电脑的本地系统时间。

  不可与下列选项同时启用 \\ \texttt{-$ $-time-match-local}, \\ \texttt{-$ $-time-match-real} 

\item{\texttt{-$ $-time-match-local}, \texttt{-$ $-time-match-real}}

  默认是 {-$ $-time-match-real}:模拟器会读取系统时钟并使用。当你的虚拟飞行与你电脑在同一个时区时,这会非常合适,因为时钟是同步的。然而,当你飞到世界的另一端,就不一样了,因为和虚拟飞行和你实际时间之间有几个小时的时差。

{-$ $-time-match-local} 选项可以解决这个系统本地时间与虚拟飞行的时差问题,将你的本地系统时间与虚拟飞行的时间同步。

 不可与 \texttt{-$ $-start-date-gmt}, \texttt{-$ $-start-date-lat}, \texttt{-$ $-start-date-sys} 选项同时启用。

\item{\texttt{-$ $-time-offset={\it [+-]hh:mm:ss}}}

  指定一个相对于之前时间选项的时间偏移量。

\item{\texttt{-$ $-timeofday={\it param}}}

  设定一天中的时段。可以用的值有 \texttt{real}, \texttt{dawn}, \texttt{morning},
  \texttt{noon}, \texttt{afternoon}, \texttt{dusk}, \texttt{evening}, \texttt{midnight}。
\end{itemize}
}
\fi
\iffalse
{
  %%%%%%%%%%%%%%%%%%%%%%%%%%%%%%%%%%%%%%%%%%%%%%%%%%%%%%%%%%%%%%%%%%%%%%%%%%%%%%%%%%%%%%%%%%%%%%%
  \subsection{Time Options}\index{time options}\index{options!time}
  %%%%%%%%%%%%%%%%%%%%%%%%%%%%%%%%%%%%%%%%%%%%%%%%%%%%%%%%%%%%%%%%%%%%%%%%%%%%%%%%%%%%%%%%%%%%%%%
  \begin{itemize}

  \item{\texttt{-$ $-enable-clock-freeze}, \texttt{-$ $-disable-clock-freeze}}

  Control whether time advances normally or is frozen.

  \item{\texttt{-$ $-start-date-gmt={\it yyyy:mm:dd:hh:mm:ss}},
  \texttt{-$ $-start-date-lat={\it yyyy:mm:dd:hh:mm:ss}},
  \texttt{-$ $-start-date-sys={\it yyyy:mm:dd:hh:mm:ss}}}

  Specify the exact startup time/date. The three functions differ in that they
  take either Greenwich Mean Time, the local time of your virtual flight, or
  your computer system's local time as the reference point.

  Incompatible with \texttt{-$ $-time-match-local}, \texttt{-$ $-time-match-real}.

  \item{\texttt{-$ $-time-match-local}, \texttt{-$ $-time-match-real}}

  {-$ $-time-match-real}, is default: Simulator time is read from the system clock, and
  is used as is. When your virtual flight is in the same timezone as where your computer
  is located, this may be desirable, because the clocks are synchronized. However,
  when you fly in a different part of the world, it may not be the case, because there
  is a number of hours difference, between the position of your computer and the position of your
  virtual flight.

  The option {-$ $-time-match-local} takes care of this by computing the timezone
  difference between your real world time zone and the position of your virtual
  flight, and local clocks are synchronized.

  Incompatible with \texttt{-$ $-start-date-gmt}, \texttt{-$ $-start-date-lat}, \texttt{-$ $-start-date-sys}.

  \item{\texttt{-$ $-time-offset={\it [+-]hh:mm:ss}}}

  Specify a time offset relative to one of the time options above.

  \item{\texttt{-$ $-timeofday={\it param}}}

  Set the time of day. Valid parameters are \texttt{real}, \texttt{dawn}, \texttt{morning},
  \texttt{noon}, \texttt{afternoon}, \texttt{dusk}, \texttt{evening}, \texttt{midnight}.

  \end{itemize}
}
\fi

\IfLanguageName{italian}{
  %%%%%%%%%%%%%%%%%%%%%%%%%%%%%%%%%%%%%%%%%%%%%%%%%%%%%%%%%%%%%%%%%%%%%%%%%%%%%%%%%%%%%%%%%%%%%%%
  \subsection{Opzioni di rete}
  \index{Opzioni di rete}\index{Opzioni!rete}
  %%%%%%%%%%%%%%%%%%%%%%%%%%%%%%%%%%%%%%%%%%%%%%%%%%%%%%%%%%%%%%%%%%%%%%%%%%%%%%%%%%%%%%%%%%%%%%%
  \begin{itemize}
  \item{\texttt{-$ $-multiplay={\it dir,Hz,host,porta}}, \texttt{-$ $-callsign={\it NOME}}}

  Imposta le opzioni per il multigiocatore (Vedi sezione \ref{multiplayer}).

  \item{\texttt{-$ $-httpd={\it porta}}, \texttt{-$ $-telnet={\it port}}}

  Abilita un \Index{server http} o un \Index{server telnet} sulla porta specificata per
  consentire l'accesso alla struttura di propriet\`{a}.

  \item{\texttt{-$ $-jpg-httpd={\it porta}}}

  Abilita l'acquisizione dello schermo da parte del server http montato sulla porta specificata.

  \item{\texttt{-$ $-proxy={\it [user:password@]host:porta}}}

  Imposta il proxy da utilizzare per le connessioni.

  \end{itemize}
}
\ifchinese
{
%%%%%%%%%%%%%%%%%%%%%%%%%%%%%%%%%%%%%%%%%%%%%%%%%%%%%%%%%%%%%%%%%%%%%%%%%%%%%%%%%%%%%%%%%%%%%%%
  \subsection{网络选项}\index{网络选项}\index{选项!网络}
  %%%%%%%%%%%%%%%%%%%%%%%%%%%%%%%%%%%%%%%%%%%%%%%%%%%%%%%%%%%%%%%%%%%%%%%%%%%%%%%%%%%%%%%%%%%%%%%
\begin{itemize}
  \item{\texttt{-$ $-multiplay={\it dir,Hz,host,port}}, \texttt{-$ $-callsign={\it ABCD}}}

  多人游戏的选项和航空器呼号。查看 \ref{multiplayer} 节。

\item{\texttt{-$ $-httpd={\it port}}, \texttt{-$ $-telnet={\it port}}}

 在特定的端口启用\Index{HTTP 服务器}或者\Index{Telnet 服务器},以便访问属性树。

\item{\texttt{-$ $-jpg-httpd={\it port}}}

 在特定的端口上启用屏幕截图 HTTP 服务器。

\item{\texttt{-$ $-proxy={\it [user:password@]host:port}}}

 指定要使用的代理服务器。
\end{itemize}
}
\fi
\iffalse
{
  %%%%%%%%%%%%%%%%%%%%%%%%%%%%%%%%%%%%%%%%%%%%%%%%%%%%%%%%%%%%%%%%%%%%%%%%%%%%%%%%%%%%%%%%%%%%%%%
  \subsection{Network Options}\index{network options}\index{options!network}
  %%%%%%%%%%%%%%%%%%%%%%%%%%%%%%%%%%%%%%%%%%%%%%%%%%%%%%%%%%%%%%%%%%%%%%%%%%%%%%%%%%%%%%%%%%%%%%%
  \begin{itemize}
  \item{\texttt{-$ $-multiplay={\it dir,Hz,host,port}}, \texttt{-$ $-callsign={\it ABCD}}}

  Set multiplay options and aircraft call-sign. See Section \ref{multiplayer}.

  \item{\texttt{-$ $-httpd={\it port}}, \texttt{-$ $-telnet={\it port}}}

  Enable \Index{http server} or \Index{telnet server} on the specified port to provide
  access to the property tree.

  \item{\texttt{-$ $-jpg-httpd={\it port}}}

  Enable screen shot http server on the specified port.

  \item{\texttt{-$ $-proxy={\it [user:password@]host:port}}}

  Specify a proxy server to use.

  \end{itemize}
}
\fi

\IfLanguageName{italian}{
  %%%%%%%%%%%%%%%%%%%%%%%%%%%%%%%%%%%%%%%%%%%%%%%%%%%%%%%%%%%%%%%%%%%%%%%%%%%%%%%%%%%%%%%%%%%%%%%
  \subsection{Opzioni percorso/tappe}
  \index{Opzioni percorso/tappe}\index{Opzioni!percorso/tappe}
  %%%%%%%%%%%%%%%%%%%%%%%%%%%%%%%%%%%%%%%%%%%%%%%%%%%%%%%%%%%%%%%%%%%%%%%%%%%%%%%%%%%%%%%%%%%%%%%
  \begin{itemize}
  \item{\texttt{-$ $-wp={\it ID[@alt]}}}

    Permette di specificare un waypoint (punto di passaggio) per il pilota automatico, \`{e} possibile
    specificare pi\`{u} waypoints (e quindi tracciare un percorso) tramite pi\`{u} istanze di questo comando.

  \item{\texttt{-$ $-flight-plan={\it [file]}}}

    Questo \`{e} il comando pi\`{u} comodo da usare per indicare al programma il proprio piano di volo
    se si ha gi\`{a} un file contenente le diverse tappe.
  \end{itemize}
}

\ifchinese
{
%%%%%%%%%%%%%%%%%%%%%%%%%%%%%%%%%%%%%%%%%%%%%%%%%%%%%%%%%%%%%%%%%%%%%%%%%%%%%%%%%%%%%%%%%%%%%%%
  \subsection{航路/导航点选项}\index{选项!航路}\index{选项!导航点}
  %%%%%%%%%%%%%%%%%%%%%%%%%%%%%%%%%%%%%%%%%%%%%%%%%%%%%%%%%%%%%%%%%%%%%%%%%%%%%%%%%%%%%%%%%%%%%%%
\begin{itemize}
  \item{\texttt{-$ $-wp={\it ID[@alt]}}}

  允许给 GC 自动驾驶指定一个导航点;也可以通过此选项的多个实例来指定多个导航点(比如一条航路)。

\item{\texttt{-$ $-flight-plan={\it [file]}}}

  如果你有多个导航点这个更加合适。你可以指定一个文件来载入。

\end{itemize}
}
\fi
\iffalse
{
  %%%%%%%%%%%%%%%%%%%%%%%%%%%%%%%%%%%%%%%%%%%%%%%%%%%%%%%%%%%%%%%%%%%%%%%%%%%%%%%%%%%%%%%%%%%%%%%
  \subsection{Route/Waypoint Options}\index{options!route}\index{options!waypoint}
  %%%%%%%%%%%%%%%%%%%%%%%%%%%%%%%%%%%%%%%%%%%%%%%%%%%%%%%%%%%%%%%%%%%%%%%%%%%%%%%%%%%%%%%%%%%%%%%
  \begin{itemize}
  \item{\texttt{-$ $-wp={\it ID[@alt]}}}

    Allows specifying a waypoint for the GC autopilot; it is possible to
    specify multiple waypoints (i.e. a route) via multiple instances of
    this command.

  \item{\texttt{-$ $-flight-plan={\it [file]}}}

    This is more comfortable if you have several waypoints. You can
    specify a file to read them from.
  \end{itemize}
}
\fi

\IfLanguageName{italian}{
  %%%%%%%%%%%%%%%%%%%%%%%%%%%%%%%%%%%%%%%%%%%%%%%%%%%%%%%%%%%%%%%%%%%%%%%%%%%%%%%%%%%%%%%%%%%%%%%
  \subsection{Opzioni IO}
  \index{Opzioni IO}\index{Opzioni!IO}
  %%%%%%%%%%%%%%%%%%%%%%%%%%%%%%%%%%%%%%%%%%%%%%%%%%%%%%%%%%%%%%%%%%%%%%%%%%%%%%%%%%%%%%%%%%%%%%%

  NB: Queste opzioni sono piuttosto orientate all'utente avanzato che sa quello che sta facendo.

  Descrizioni pi\`{u} dettagliate dei vari parametri IO possono essere trovate nel file README.IO
  presente all'interno della directory Docs nella cartella d'installazione di \FlightGear{}.

  \begin{itemize}

  \item{\texttt{-$ $-atlas={\it parametri}}}

    Avvia la connessione utilizzando il protocollo Atlas (utilizzato da Atlas e TerraSync).

  \item{\texttt{-$ $-atcsim={\it parametri}}}

    Apre la connessione utilizzando il protocollo ATC Sim (atc610x).

  \item{\texttt{-$ $-AV400={\it parametri}}}

    Avvia il collegamento usando un Garmin  serie 196/296 GPS.

  \item{\texttt{-$ $-AV400Sim={\it parametri}}}

    Avvia il collegamento usando un Garmin serie 400 GPS.

  \item{\texttt{-$ $-generic={\it parametri}}}

    Apre la connessione utilizzando il protocollo generico (definito da un file XML).

  \item{\texttt{-$ $-garmin={\it parametri}}}

    Apre la connessione utilizzando il protocollo GPS Garmin.

  \item{\texttt{-$ $-joyclient={\it parametri}}}

    Apre il collegamento ad un joystick Agwagon.

  \item{\texttt{-$ $-jsclient={\it parametri}}}

    Apre il collegamento ad un joystick remoto.

  \item{\texttt{-$ $-native-ctrls={\it parametri}}}

    Apre la connessione utilizzando il protocollo FG Native Controls.

  \item{\texttt{-$ $-native-fdm={\it parametri}}}

    Apre la connessione utilizzando il protocollo FG Native FDM.

  \item{\texttt{-$ $-native-gui={\it parametri}}}

    Apre la connessione utilizzando il protocollo FG Native GUI.

  \item{\texttt{-$ $-native={\it parametri}}}

    Apre la connessione utilizzando il protocollo FG Native.

  \item{\texttt{-$ $-nmea={\it parametri}}}

    Apre la connessione utilizzando il protocollo NMEA.

  \item{\texttt{-$ $-opengc={\it parametri}}}

    Apre la connessione utilizzando il protocollo OpenGC.

  \item{\texttt{-$ $-props={\it parametri}}}

    Apre la connessione utilizzando il gestore di propriet\`{a} interattivo.

  \item{\texttt{-$ $-pve={\it parametri}}}

    Apre la connessione utilizzando il protocollo PVE.

  \item{\texttt{-$ $-ray={\it parametri}}}

    Apre la connessione utilizzando il protocollo RayWoodworth motion chair.

  \item{\texttt{-$ $-rul={\it parametri}}}

    Apre la connessione utilizzando il protocollo RUL.

  \end{itemize}
}
\ifchinese
{
 %%%%%%%%%%%%%%%%%%%%%%%%%%%%%%%%%%%%%%%%%%%%%%%%%%%%%%%%%%%%%%%%%%%%%%%%%%%%%%%%%%%%%%%%%%%%%%%
  \subsection{IO 选项}\index{选项!IO}
  %%%%%%%%%%%%%%%%%%%%%%%%%%%%%%%%%%%%%%%%%%%%%%%%%%%%%%%%%%%%%%%%%%%%%%%%%%%%%%%%%%%%%%%%%%%%%%%
注意:这些选项面向高级用户,明白自己做什么。

更多有关 IO 参数的细节描述在你 \FlightGear{} 安装的 Docs 目录的 README.IO 文件里能找到。

\begin{itemize}

  \item{\texttt{-$ $-atlas={\it params}}}

  使用 Atlas 协议打开连接(用于 Atlas 和 TerraSync)

 \item{\texttt{-$ $-atcsim={\it params}}}

  使用 ATC Sim 协议打开连接(atc610x)。

 \item{\texttt{-$ $-AV400={\it params}}}

  打开连接以驱动一个 Garmin 196/296 系列 GPS。 

 \item{\texttt{-$ $-AV400Sim={\it params}}}

  打开连接以驱动一个 Garmin 400 系列 GPS。

 \item{\texttt{-$ $-generic={\it params}}}

  使用通用(XML定义)的协议打开连接。

 \item{\texttt{-$ $-garmin={\it params}}}
   
  使用 Garmin GPS 协议打开连接。

 \item{\texttt{-$ $-joyclient={\it params}}}
  
  连接 Agwagon 游戏杆。

 \item{\texttt{-$ $-jsclient={\it params}}}

  连接远程游戏杆。

 \item{\texttt{-$ $-native-ctrls={\it params}}

  使用 FG 原生控制协议打开连接。

 \item{\texttt{-$ $-native-fdm={\it params}}}

  使用 FG 原生 FDM 协议打开连接。
 
 \item{\texttt{-$ $-native-gui={\it params}}}

  使用 FG 原生图形协议打开连接。

 \item{\texttt{-$ $-native={\it params}}}

  使用 FG 原生协议打开连接。

 \item{\texttt{-$ $-nmea={\it params}}}

  使用 NMEA 协议打开连接。

 \item{\texttt{-$ $-opengc={\it params}}}

  使用 OpenGC 协议打开连接。

 \item{\texttt{-$ $-props={\it params}}}

  连接非活动的属性管理器。

 \item{\texttt{-$ $-pve={\it params}}}

  使用 PVE 打开连接。

 \item{\texttt{-$ $-ray={\it params}}}

  使用 RayWoodworth 全动式座椅协议打开连接。

 \item{\texttt{-$ $-rul={\it params}}}

  使用 RUL 协议打开连接。
  
  \end{itemize}
}
\fi
\iffalse
{
  %%%%%%%%%%%%%%%%%%%%%%%%%%%%%%%%%%%%%%%%%%%%%%%%%%%%%%%%%%%%%%%%%%%%%%%%%%%%%%%%%%%%%%%%%%%%%%%
  \subsection{IO Options}\index{options!IO}
  %%%%%%%%%%%%%%%%%%%%%%%%%%%%%%%%%%%%%%%%%%%%%%%%%%%%%%%%%%%%%%%%%%%%%%%%%%%%%%%%%%%%%%%%%%%%%%%

  NB: These options are rather geared to the advanced user who knows what he is doing.

  More detailed descriptions of the various IO parameters can be found in the README.IO file
  within the Docs directory of your \FlightGear{} installation.

  \begin{itemize}

  \item{\texttt{-$ $-atlas={\it params}}}

    Open connection using the Atlas protocol (used by Atlas and TerraSync).

  \item{\texttt{-$ $-atcsim={\it params}}}

    Open connection using the ATC Sim protocol (atc610x).

  \item{\texttt{-$ $-AV400={\it params}}}

    Open connection to drive a Garmin 196/296 series GPS

  \item{\texttt{-$ $-AV400Sim={\it params}}}

    Open connection to drive a Garmin 400 series GPS

  \item{\texttt{-$ $-generic={\it params}}}

    Open connection using the Generic (XML-defined) protocol.

  \item{\texttt{-$ $-garmin={\it params}}}

    Open connection using the Garmin GPS protocol.

  \item{\texttt{-$ $-joyclient={\it params}}}

    Open connection to an Agwagon joystick.

  \item{\texttt{-$ $-jsclient={\it params}}}

    Open connection to a remote joystick.

  \item{\texttt{-$ $-native-ctrls={\it params}}}

    Open connection using the FG native Controls protocol.

  \item{\texttt{-$ $-native-fdm={\it params}}}

    Open connection using the FG Native FDM protocol.

  \item{\texttt{-$ $-native-gui={\it params}}}

    Open connection using the FG Native GUI protocol.

  \item{\texttt{-$ $-native={\it params}}}

    Open connection using the FG Native protocol.

  \item{\texttt{-$ $-nmea={\it params}}}

    Open connection using the NMEA protocol.

  \item{\texttt{-$ $-opengc={\it params}}}

    Open connection using the OpenGC protocol.

  \item{\texttt{-$ $-props={\it params}}}

    Open connection using the interactive property manager.

  \item{\texttt{-$ $-pve={\it params}}}

    Open connection using the PVE protocol.

  \item{\texttt{-$ $-ray={\it params}}}

    Open connection using the RayWoodworth motion chair protocol.

  \item{\texttt{-$ $-rul={\it params}}}

    Open connection using the RUL protocol.

  \end{itemize}
}
\fi

\IfLanguageName{italian}{
  %%%%%%%%%%%%%%%%%%%%%%%%%%%%%%%%%%%%%%%%%%%%%%%%%%%%%%%%%%%%%%%%%%%%%%%%%%%%%%%%%%%%%%%%%%%%%%%
  \subsection{Opzioni di debug}
  \index{Opzioni di debug}\index{Opzioni!debug}
  %%%%%%%%%%%%%%%%%%%%%%%%%%%%%%%%%%%%%%%%%%%%%%%%%%%%%%%%%%%%%%%%%%%%%%%%%%%%%%%%%%%%%%%%%%%%%%%

  NB: Queste opzioni sono piuttosto orientate all'utente avanzato che sa quello che sta facendo.

  \begin{itemize}

  \item{\texttt{-$ $-enable-fpe}}

  Abilita l'interruzione del simulatore al verificarsi di un errore di virgola mobile (Exception Floating Point).

  \item{\texttt{-$ $-fgviewer}}

  Invece di caricare l'intera simulazione, caricare una visualizzazione leggera OSG. Utile per controllare i modelli di aeromobili.

  \item{\texttt{-$ $-log-level={\it livello}}}

  Imposta il livello di registrazione dei log. I valori validi sono: \texttt{bulk}, \texttt{debug}, \texttt{info}, \texttt{warn}, \texttt{alert}.

  \item{\texttt{-$ $-prop:[tip:]nome=valore}}

  Assegna un determinato valore ad una propriet\`{a}. Si pu\`{o} opzionalmente
  specificare il tipo di propriet\`{a} (\texttt{double}, \texttt{string}, \texttt{boolean}).

  Esempio: \texttt{-$ $-prop:/engines/engine[0]/running=true} starts the simulator with running engines.

  Altro esempio:

  \texttt{-$ $-aircraft=c172p}\\
  \texttt{-$ $-prop:/consumables/fuels/tank[0]/level-gal=10}\\
  \texttt{-$ $-prop:/consumables/fuels/tank[1]/level-gal=10}\\

  (riempie il Cessna di carburante per un breve volo).

  \item{\texttt{-$ $-trace-read={\it parametri}}}

  Traccia la legge per una propriet\`{a}, sono ammesse pi\`{u} istanze.

  \item{\texttt{-$ $-trace-write={\it parametri}}}

  Traccia le scritture di una propriet\`{a}, sono ammesse pi\`{u} istanze.
  \end{itemize}
}

\ifchinese
{
 %%%%%%%%%%%%%%%%%%%%%%%%%%%%%%%%%%%%%%%%%%%%%%%%%%%%%%%%%%%%%%%%%%%%%%%%%%%%%%%%%%%%%%%%%%%%%%%
  \subsection{调式选项}\index{选项!调式}
  %%%%%%%%%%%%%%%%%%%%%%%%%%%%%%%%%%%%%%%%%%%%%%%%%%%%%%%%%%%%%%%%%%%%%%%%%%%%%%%%%%%%%%%%%%%%%%%
注意:这些选项仅限那些懂得自己在做什么的高级用户使用。

  \begin{itemize}

  \item{\texttt{-$ $-enable-fpe}}
 
   启用浮点数异常时中止。

  \item{\texttt{-$ $-fgviewer}}
  
   与载入整个模拟器不同,只载入轻量级的 OSG 查看器。用来检查航空器建模。

  \item{\texttt{-$ $-log-level={\it LEVEL}}}

   设定日志记录级别。可用的值有 \texttt{bulk}, \texttt{debug}, \texttt{info}, \texttt{warn}, \texttt{alert}。

  \item{\texttt{-$ $-prop:[type:]name=value}}

   设定属性,把值 \texttt{value} 赋给  \texttt{name}。

  例如:\texttt{-$ $-prop:/engines/engine[0]/running=true} 启动模拟器时引擎已经运转。

  又例如要为赛斯纳短途飞行加油。:

  \texttt{-$ $-aircraft=c172p}\\
  \texttt{-$ $-prop:/consumables/fuels/tank[0]/level-gal=10}\\
  \texttt{-$ $-prop:/consumables/fuels/tank[1]/level-gal=10}\\

  你还可以指定属性类型 (\texttt{double}, \texttt{string}, \texttt{boolean})。

 \item{\texttt{-$ $-trace-read={\it params}}}
  
   追踪一个属性值的读取;可以使用多个选项。

  \item{\texttt{-$ $-trace-write={\it params}}}
   
   追踪一个属性值的写入;可以使用多个选项。
\end{itemize}
}

\fi
\iffalse
{

  %%%%%%%%%%%%%%%%%%%%%%%%%%%%%%%%%%%%%%%%%%%%%%%%%%%%%%%%%%%%%%%%%%%%%%%%%%%%%%%%%%%%%%%%%%%%%%%
  \subsection{Debugging options}\index{options!debugging}
  %%%%%%%%%%%%%%%%%%%%%%%%%%%%%%%%%%%%%%%%%%%%%%%%%%%%%%%%%%%%%%%%%%%%%%%%%%%%%%%%%%%%%%%%%%%%%%%

  NB: These options are rather geared to the advanced user who knows what he is doing.

  \begin{itemize}

  \item{\texttt{-$ $-enable-fpe}}

  Enable abort on a Floating Point Exception.

  \item{\texttt{-$ $-fgviewer}}

  Instead of loading the entire simulator, load a lightweight OSG viewer. Useful for checking aircraft
  models.

  \item{\texttt{-$ $-log-level={\it LEVEL}}}

  Set the logging level. Valid values are \texttt{bulk}, \texttt{debug}, \texttt{info}, \texttt{warn}, \texttt{alert}.

  \item{\texttt{-$ $-prop:[type:]name=value}}

  Set property \texttt{name} to \texttt{value}

  Example: \texttt{-$ $-prop:/engines/engine[0]/running=true} starts the simulator with running engines.

  Another example:

  \texttt{-$ $-aircraft=c172p}\\
  \texttt{-$ $-prop:/consumables/fuels/tank[0]/level-gal=10}\\
  \texttt{-$ $-prop:/consumables/fuels/tank[1]/level-gal=10}\\

  fills the Cessna for a short flight. You may optionally specific
  the property type (\texttt{double}, \texttt{string}, \texttt{boolean}).

  \item{\texttt{-$ $-trace-read={\it params}}}

  Trace the reads for a property; multiple instances are allowed.

  \item{\texttt{-$ $-trace-write={\it params}}}

  Trace the writes for a property; multiple instances are allowed.
  \end{itemize}
}
\fi

\IfLanguageName{italian}{
  %%%%%%%%%%%%%%%%%%%%%%%%%%%%%%%%%%%%%%%%%%%%%%%%%%%%%%%%%%%%%%%%%%%%%%%%%%%%%%%%%%%%%%%%%%%%%%%
  \section{Supporto per il joystick\label{joysticksupp}}\index{joystick}
  %%%%%%%%%%%%%%%%%%%%%%%%%%%%%%%%%%%%%%%%%%%%%%%%%%%%%%%%%%%%%%%%%%%%%%%%%%%%%%%%%%%%%%%%%%%%%%%
  Riuscite a immaginare un pilota nel suo Cessna che lo controlla soltanto con una tastiera? Per ottenere la
  giusta sensazione di volo servono un joystick e una pedaliera!

  \FlightGear{} ha integrato il supporto per il joystick, che rileva automaticamente qualsiasi joystick o
  pedaliera collegati. Basta collegare la periferica e avviare il simulatore.

  Si pu\`{o} vedere la configurazione per l'utilizzo del joystick andando nel men\`{u} Guida-->Configurazione Joystick.
  Questa finestra mostra il nome del vostro joystick, e la funzione abbinata a ciascun pulsante/asse di controllo.
  \`{e} possibile premere un tasto o muovere il joystick per vedere esattamente la sua funzione e modificarla. In
  caso di modifiche, la configurazione ha effetto immediato e sar\`{a} salvata per tutti i prossimi voli.

  Se si dispone di un joystick comune, ci sono molte probabilit\`{a} che qualcuno abbia gi\`{a} istituito una
  configurazione specifica di FlightGear per esso, e si pu\`{o} semplicemente iniziare a volare!.

}
\ifchinese
{
 %%%%%%%%%%%%%%%%%%%%%%%%%%%%%%%%%%%%%%%%%%%%%%%%%%%%%%%%%%%%%%%%%%%%%%%%%%%%%%%%%%%%%%%%%%%%%%%
  \section{游戏杆支持\label{joysticksupp}}\index{Joystick 游戏杆}
  %%%%%%%%%%%%%%%%%%%%%%%%%%%%%%%%%%%%%%%%%%%%%%%%%%%%%%%%%%%%%%%%%%%%%%%%%%%%%%%%%%%%%%%%%%%%%%%
你能想象一名飞行员只用键盘来控制塞斯纳飞机吗?要想获得更好的飞行体验,你需要一把飞行游戏杆/驾驶盘,外加脚蹬踏板。

\FlightGear{} 已经集成了飞行游戏杆支持,会自动检测任何连接的游戏杆、驾驶盘或脚踏板。只需要连接上你的游戏杆并启动模拟器即可。

通过菜单 Help -> Joystick Configuration 你可以看到 \FlightGear{} 已经配置好了你的游戏杆。这个对话框显示了游戏杆的名字以及每个按键和轴的设置。你可以按一个按键或者移动游戏杆来查看具体的映射。

如果你有一把常见游戏杆,也许有人已经在 FlightGear 里面为它设定好了,你进入游戏就可以飞了!如果你想改变某一个按键或轴的配置,只需要用游戏杆配置对话框来编辑即可。

如果你的游戏杆不太常见,那么 \FlightGear{} 将默认使用一个简单的游戏杆配置。若你想改变配置,只需要用游戏杆配置对话框来编辑即可。配置会即刻起效,并一直保存到你下次飞行。
}
\fi
\iffalse
{
  %%%%%%%%%%%%%%%%%%%%%%%%%%%%%%%%%%%%%%%%%%%%%%%%%%%%%%%%%%%%%%%%%%%%%%%%%%%%%%%%%%%%%%%%%%%%%%%
  \section{Joystick support\label{joysticksupp}}\index{joystick}
  %%%%%%%%%%%%%%%%%%%%%%%%%%%%%%%%%%%%%%%%%%%%%%%%%%%%%%%%%%%%%%%%%%%%%%%%%%%%%%%%%%%%%%%%%%%%%%%
  Could you imagine a pilot in his or her Cessna controlling the machine with
  a keyboard alone? For getting the proper feeling of flight you will need a
  joystick/yoke plus rudder pedals.

  \FlightGear{} has integrated joystick support, which automatically detects
  any joystick, yoke, or pedals attached. Simply plug in your joystick and start
  the simulator.

  You can see how \FlightGear{} has configured your joystick by selecting
  Help -> Joystick Configuration from the menu.  This dialog shows the name
  of your joystick, and what each of the buttons and control axes have been set to.
  You can press a button or move the joystick to see exactly what control it maps to.

  If you have a common joystick, there's every chance that someone has already
  set up FlightGear specific configuration for it, and you can simply go and fly!
  If you wish to change the configuration of a particular button/axis, simply edit
  it using the Joystick Configuration dialog.

  If your joystick is more unusual, then \FlightGear{} will by default use a simple
  joystick configuration for it.  To change the configuration, simply use the Joystick
  Configuration dialog to select what you wish each button or movement to do.  The
  configuration takes effect immediately and will be saved for your next flight.

}
\fi
%% Revision 0.02  1998/09/29  bernhard
%% revision 0.10  1998/10/01  michael
%% final proofreading for release
%% revision 0.11  1998/11/01  michael
%% Added pic from Arizona takeoff
%% revision 0.20  1999/06/04  michael
%% added new options
%% revision 0.3 2000/04/20 michael
%% added numerous new options (number rapidly growing...)
%% revision 0.4 2001/05/12 michael
%% description of .fgfsrc/system.fgfsrc
%% again many new options
%% joystick section added based on John Check's joystick howto
%% updated arizona pic
%% revision 0.4 2001/07/01 martin
%% comments on debug options under Unix
%% revision 0.5 2002/01/01 michael/martin
%% added more options, notably the --aircraft option
%% tweaks in the introductory section
%% revised joystick section based on fgjs
%% New KFSO picture
%% Paragraphs on UIUC + tweaks by Martin
%% Added Section on IIO options
%% revision 0.6 2002/09/05 michael
%% Corrected win shell call in 4.2
%% added/corrected/deleted several command line options
%% Added large section on built-in joystick support by Dave Perry
%% Added section/table with complete list of aircraft
%% Command line options in quotes for Win batch start
%% D Perry edits:  p.42 changed feed to feet, p43. changed incentive to challenging
%% p.51 added a space after USB in the joystick name per what jstest really reports.
%% p.52 removed at after evaluating, p.53 added "and change this to" along with the LaTex syntax between <axis n="2"> and <axis n="1">
%% p.53 added a sentence pointing out that Windows and Linux often assign different axes and buttons to functions and the above edit procedure easily allows moving from one OS to antother.
%% Added section on env. vars, the Joystick Information menu option and aircraft.
%% Revision 16/10/08: corrected various grammar, syntax and spelling errors.
%% Revision 22/12/09: Sync with actual command-line options, significant text update.

