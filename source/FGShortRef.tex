%%
%% FGShortRef.tex -- Flight Gear documentation: Short reference
%%
%% Written by Michael Basler, starting May 2001.
%%
%% Copyright (C) 2002 Michael Basler
%%
%% This program is free software; you can redistribute it and/or
%% modify it under the terms of the GNU General Public License as
%% published by the Free Software Foundation; either version 2 of the
%% License, or (at your option) any later version.
%%
%% This program is distributed in the hope that it will be useful, but
%% WITHOUT ANY WARRANTY; without even the implied warranty of
%% MERCHANTABILITY or FITNESS FOR A PARTICULAR PURPOSE.  See the GNU
%% General Public License for more details.
%%
%% You should have received a copy of the GNU General Public License
%% along with this program; if not, write to the Free Software
%% Foundation, Inc., 675 Mass Ave, Cambridge, MA 02139, USA.
%%
%% $Id: FGShortRef.tex,v 0.6 2002/09/09 michael
%% (Log is kept at end of this file)

\documentclass[10pt]{article}
\usepackage{graphicx}
\usepackage{times}
\usepackage{hyperref}
\usepackage{multicol}
\pagestyle{empty}
\usepackage{a4}
\usepackage[german,french,english]{babel}
\selectlanguage{english}

\newcommand{\Index}[1]{#1\index{#1}}
\newcommand{\FlightGear}{{\itshape\bfseries FlightGear}}
\newcommand{\TerraGear}{{\itshape\bfseries TerraGear}}
\newcommand{\SimGear}{{\itshape\bfseries SimGear}}
\newcommand{\PLIB}{{\itshape\bfseries PLIB}}
\newcommand{\JSBSim}{{\itshape\bfseries JSBSim}}
\newcommand{\web}[1]{\href{#1}{#1}}
\newcommand{\mail}[1]{\href{mailto:#1}{#1}}
\newcommand{\Cygwin}{{\itshape\bfseries Cygwin}}

\newcommand{\longpage}{\enlargethispage{\baselineskip}}
\newcommand{\shortpage}{\enlargethispage{-\baselineskip}}

\makeindex

\begin{document}
\longpage

\centerline{\large \textbf{\FlightGear{} Short Reference}}
\medskip

\scriptsize \noindent
%\footnotesize \noindent
 \FlightGear{} is a free flight simulator developed collectively over the
 Internet under the GPL.  For more information see  \web{http://www.flightgear.org/}\\

\hspace*{-8mm}
\begin{tabular}{ll}
\textbf{Program Start:}  & Linux/UNIX via fgfs under FlightGear/,\\
                         & Mac OS X via FlightGear.app under /Applications/,\\
                         & Windows via the \FlightGear{} wizard fgrun.exe under
         $\backslash$Program Files$\backslash$FlightGear$\backslash$bin$\backslash$Win32$\backslash$\\

\textbf{Engine Start:}   & Set ignition switch to BOTH (``\}'' three times). Set mixture to 100\%.
                           Set throttle to about 25\%. Operate starter using the ``s'' key. \\
                         & Once the engine has started, set throttle back to idle.
                           Release parking brake (``B''), if applied.
\end{tabular}
\medskip

%%%%%%%%%%%%%%%%%%%%%%%%%%%%%%%%%%%%%%%%%%%%%%%%%%%%%%%%%%%%
 \noindent
 \textbf{Keyboard controls:}
\begin{multicols}{2}
 \noindent
Table 1: \textit{Directional controls (activated \texttt{NumLock}})\\

\noindent
/home/tonghuix/git-pro/flightgear-getstart/bin/../source/tab2.tex
\bigskip

%%%%%%%%%%%%%%%%%%%%%%%%%%%%%%%%%%%%%%%%%%%%%%%%%%%%%%%%%%%%
 \noindent
Table 2: \textit{Engine controls}
\medskip

 \noindent
%%
%% tab7.tex -- Flight Gear documentation: Installation and Getting Started
%% Keyboard controls table 6/Engine related controls
%%
%% Written by Michael Basler, started September 1998.
%%
%% Copyright (C) 2002 Michael Basler (pmb@epost.de)
%%
%%
%% This program is free software; you can redistribute it and/or
%% modify it under the terms of the GNU General Public License as
%% published by the Free Software Foundation; either version 2 of the
%% License, or (at your option) any later version.
%%
%% This program is distributed in the hope that it will be useful, but
%% WITHOUT ANY WARRANTY; without even the implied warranty of
%% MERCHANTABILITY or FITNESS FOR A PARTICULAR PURPOSE.  See the GNU
%% General Public License for more details.
%%
%% You should have received a copy of the GNU General Public License
%% along with this program; if not, write to the Free Software
%% Foundation, Inc., 675 Mass Ave, Cambridge, MA 02139, USA.
%%
%% $Id: tab6.tex,v 0.5 2002/15/02 michael
%% (Log is kept at end of this file)
%%%%%%%%%%%%%%%%%%%%%%%%%%%%%%%%%%%%%%%%%%%%%%%%%%%%%%%%%%%%%%%%%%%%%%%%%%%%%%%%%%%%%%%%%%%%%%%%
\begin{tabular}{|l|l|}\hline
Key      &  Action\\ \hline
   SPACE & Fire starter on selected engine(s)\\
   !     & Select 1st engine\\
   @		 & Select 2nd engine\\
  \#     & Select 3rd engine\\
  \$     & Select 4th engine\\
  \{     & Decrease Magneto on Selected Engine\\
  \}     & Increase Magneto on Selected Engine\\
  $\sim$   & Select all Engines\\\hline
\end{tabular}

%% revision 0.5 2002/02/15 michael
%% Initial revision
\medskip

%%%%%%%%%%%%%%%%%%%%%%%%%%%%%%%%%%%%%%%%%%%%%%%%%%%%%%%%%%%%
 \noindent
Table 3: \textit{Miscellaneous aircraft controls}
\medskip

 \noindent
/home/tonghuix/git-pro/flightgear-getstart/bin/../source/tab8.tex
\bigskip


%%%%%%%%%%%%%%%%%%%%%%%%%%%%%%%%%%%%%%%%%%%%%%%%%%%%%%%%%%%%
 \noindent
Table 4: \textit{General simulator controls}
\medskip

 \noindent
%%
%% tab6.tex -- Flight Gear documentation: The FlightGear Manual
%% Keyboard controls table 5/key actions for autopilot enabled
%%
%% Written by Michael Basler, started September 1998.
%%
%% Copyright (C) 2002 Michael Basler
%%
%%
%% This program is free software; you can redistribute it and/or
%% modify it under the terms of the GNU General Public License as
%% published by the Free Software Foundation; either version 2 of the
%% License, or (at your option) any later version.
%%
%% This program is distributed in the hope that it will be useful, but
%% WITHOUT ANY WARRANTY; without even the implied warranty of
%% MERCHANTABILITY or FITNESS FOR A PARTICULAR PURPOSE.  See the GNU
%% General Public License for more details.
%%
%% You should have received a copy of the GNU General Public License
%% along with this program; if not, write to the Free Software
%% Foundation, Inc., 675 Mass Ave, Cambridge, MA 02139, USA.
%%
%% $Id: tab5.tex,v 0.6 2002/09/09 michael
%% (Log is kept at end of this file)
%%%%%%%%%%%%%%%%%%%%%%%%%%%%%%%%%%%%%%%%%%%%%%%%%%%%%%%%%%%%%%%%%%%%%%%%%%%%%%%%%%%%%%%%%%%%%%%%
\begin{tabular}{|l|l|}\hline
\iflanguage{english}{
 Key          & Action\\\hline
  p           & Pause simulator \index{pause}\\
  a / A       & Simulation speed up/slow down\\
  t / T       & Clock speed up/slow down       \\
  Ctrl-R      & Instant replay \\
  F3          & Save screen shot\\
  ESC         & Exit program\\\hline
}{}
\iflanguage{french}{
 Touche       & Action\\\hline
  p           & Mettre le simulateur en pause \index{pause}\\
  a / A       & Vitesse de simulation acc\'{e}l'{e}rer/ralentir\\
  t / T       & Vitesse de l'horloge acc\'{e}l'{e}rer/ralentir\\
  Ctrl-R      & Ralenti instantan\'{e}\\
  F3          & Sauvegarder la capture d'\'{e}cran\\
  ESC         & Quitter le programme\\\hline
}{}
 \end{tabular}

%% revision 0.5 2002/02/15 michael
%% Initial revision

\medskip

%%%%%%%%%%%%%%%%%%%%%%%%%%%%%%%%%%%%%%%%%%%%%%%%%%%%%%%%%%%%
\noindent
Table 5: \textit{View controls (de-activated \texttt{NumLock})}
\medskip

 \noindent
 %%
%% tab3.tex -- Flight Gear documentation: Installation and Getting Started
%% Keyboard controls table 2/View directions
%%
%% Written by Michael Basler, started September 1998.
%%
%% Copyright (C) 2002 Michael Basler (pmb@epost.de)
%%
%%
%% This program is free software; you can redistribute it and/or
%% modify it under the terms of the GNU General Public License as
%% published by the Free Software Foundation; either version 2 of the
%% License, or (at your option) any later version.
%%
%% This program is distributed in the hope that it will be useful, but
%% WITHOUT ANY WARRANTY; without even the implied warranty of
%% MERCHANTABILITY or FITNESS FOR A PARTICULAR PURPOSE.  See the GNU
%% General Public License for more details.
%%
%% You should have received a copy of the GNU General Public License
%% along with this program; if not, write to the Free Software
%% Foundation, Inc., 675 Mass Ave, Cambridge, MA 02139, USA.
%%
%% $Id: tab2.tex,v 0.6 2002/09/09 michael
%% (Log is kept at end of this file)
%%%%%%%%%%%%%%%%%%%%%%%%%%%%%%%%%%%%%%%%%%%%%%%%%%%%%%%%%%%%%%%%%%%%%%%%%%%%%%%%%%%%%%%%%%%%%%%%
\begin{tabular}{|c|l|}\hline
 Numeric Key  &  View direction\index{view directions}\\\hline
    Shift-8 & Forward\\
    Shift-7 & Left/forward\\
    Shift-4 & Left\\
    Shift-1 & Left/back\\
    Shift-2 & Back\\
    Shift-3 & Right/back\\
    Shift-6 & Right\\
    Shift-9 & Right/forward\\\hline
\end{tabular}

%% revision 0.5 2002/02/15 michael
%% Initial revision
\medskip


%%%%%%%%%%%%%%%%%%%%%%%%%%%%%%%%%%%%%%%%%%%%%%%%%%%%%%%%%%%%
\medskip

 \noindent
 Table 6: \textit{Autopilot controls}
\medskip

\noindent
/home/tonghuix/git-pro/flightgear-getstart/bin/../source/tab5.tex
\medskip

%%%%%%%%%%%%%%%%%%%%%%%%%%%%%%%%%%%%%%%%%%%%%%%%%%%%%%%%%%%%
 \noindent
 Table 7: \textit{Display controls}
\medskip

 \noindent
%%
%% tab4.tex -- Flight Gear documentation: The FlightGear Manual
%% Keyboard controls table 3/Additional view options
%%
%% Written by Michael Basler, started September 1998.
%%
%% Copyright (C) 2002 Michael Basler
%%
%%
%% This program is free software; you can redistribute it and/or
%% modify it under the terms of the GNU General Public License as
%% published by the Free Software Foundation; either version 2 of the
%% License, or (at your option) any later version.
%%
%% This program is distributed in the hope that it will be useful, but
%% WITHOUT ANY WARRANTY; without even the implied warranty of
%% MERCHANTABILITY or FITNESS FOR A PARTICULAR PURPOSE.  See the GNU
%% General Public License for more details.
%%
%% You should have received a copy of the GNU General Public License
%% along with this program; if not, write to the Free Software
%% Foundation, Inc., 675 Mass Ave, Cambridge, MA 02139, USA.
%%
%% $Id: tab3.tex,v 0.6 2002/09/09 michael
%% (Log is kept at end of this file)
%%%%%%%%%%%%%%%%%%%%%%%%%%%%%%%%%%%%%%%%%%%%%%%%%%%%%%%%%%%%%%%%%%%%%%%%%%%%%%%%%%%%%%%%%%%%%%%%
\begin{tabular}{|l|l|}\hline
 Key              &         Action\\\hline
 P                &    Toggle \Index{instrument panel} on/off \\
 c                &    Toggle3D/2D cockpit \index{2D cockpit} (if both are available) \index{3D cockpit}\index{cockpit}\\
 S                &    Cycle panel style full/mini\\
 Shift-F5/F6      &    Shift the panel in y direction\\
 Shift-F7/F8      &    Shift the panel in x direction\\
 Shift-F3         &    Read a panel from a property list\\
 Ctrl-c           &    Toggle panel/cockpit hotspot visibility\\
 i/I              &    Minimize/maximize HUD              \\
 h/H              &    Change color  of HUD/toggle HUD off\\
                  &    forward/backward      \\   \hline
 x/X              &    Zoom in/out\\
 v/V              &    Cycle \Index{view modes} forth and back\\
 Ctrl-v           &    Reset \Index{view modes} to pilot view\\
 z/Z              &    Increase/Decrease visibility (fog) \\
 F10              &    Toggle menu on/off\\ \hline
 \end{tabular}

%% revision 0.5 2002/02/15 michael
%% Initial revision

\bigskip

\end{multicols}

 \noindent
 \textbf{Mouse controlled functions:}
 There are three mouse modes, which can be swapped between by clicking the right mouse button.

 \begin{enumerate}
 \item In \textbf{normal} mode (pointer cursor), the panel and cockpit controls can be
 operated using the mouse. To change a control, click with the left/middle mouse button
 on the corresponding knob/lever. Generally, the left side of the control decreases the setting,
 while the right side increases the setting. The left mouse button makes small changes while the
 middle button makes larger ones. The scrollwheel may be used on some controls.
 Press Ctrl-c to view panel/cockpit hotspots.

 \item In \textbf{control} mode (cross hair cursor), the mouse is used to directly control
 the aircraft in the absence of a joystick. Moving the mouse controls the aileron (left/right)
 and elevator (forwards/backwards). Holding the left mouse button down allows control of the rudder (left/right), while holding the middle mouse button controls throttle (forwards/backwards). The scrollwheel controls
 elevator trim. Using auto-coordination (\texttt{-$ $-enable-auto-coordination}) is recommended.

 \item In \textbf{view} mode (arrow cursor), you can control the view direction using the mouse.
 Clicking the left mouse button resets the view direction. Holding the middle button down while
 moving the mouse shifts the viewpoint. The scrollwheel may be used to control the field of view.

\end{enumerate}

 \noindent
 Short Reference by M. Basler, S. Buchanan for \FlightGear{} version 2.10.0.\\
 Published under the GPL (\web{http://www.gnu.org/copyleft/gpl.html})

\end{document}

%% Revision 0.4  2001/01/05  michael
%% Initial revision based on getting Started 0.4
%% Revision 0.5 2002/02/15   michael
%% based on \inputs of revised table files from Getting Started 0.5
