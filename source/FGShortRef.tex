%%
%% FGShortRef.tex -- Flight Gear documentation: Short reference
%%
%% Written by Michael Basler, starting May 2001.
%%
%% Copyright (C) 2002 Michael Basler  (pmb@epost.de)
%%
%% This program is free software; you can redistribute it and/or
%% modify it under the terms of the GNU General Public License as
%% published by the Free Software Foundation; either version 2 of the
%% License, or (at your option) any later version.
%%
%% This program is distributed in the hope that it will be useful, but
%% WITHOUT ANY WARRANTY; without even the implied warranty of
%% MERCHANTABILITY or FITNESS FOR A PARTICULAR PURPOSE.  See the GNU
%% General Public License for more details.
%%
%% You should have received a copy of the GNU General Public License
%% along with this program; if not, write to the Free Software
%% Foundation, Inc., 675 Mass Ave, Cambridge, MA 02139, USA.
%%
%% $Id: FGShortRef.tex,v 0.6 2002/09/09 michael
%% (Log is kept at end of this file)

\documentclass[10pt]{article}
\usepackage{graphicx}
\usepackage{times}
\usepackage{hyperref}
\usepackage{multicol}
\pagestyle{empty}
\usepackage{a4}

\newcommand{\Index}[1]{#1\index{#1}}
\newcommand{\FlightGear}{{\itshape\bfseries FlightGear}}
\newcommand{\TerraGear}{{\itshape\bfseries TerraGear}}
\newcommand{\SimGear}{{\itshape\bfseries SimGear}}
\newcommand{\PLIB}{{\itshape\bfseries PLIB}}
\newcommand{\JSBSim}{{\itshape\bfseries JSBSim}}
\newcommand{\web}[1]{\href{#1}{#1}}
\newcommand{\mail}[1]{\href{mailto:#1}{#1}}
\newcommand{\Cygwin}{{\itshape\bfseries Cygwin}}

\newcommand{\longpage}{\enlargethispage{\baselineskip}}
\newcommand{\shortpage}{\enlargethispage{-\baselineskip}}

\makeindex

\begin{document}
\longpage

\centerline{\large \textbf{\FlightGear{} Short Reference}}
\medskip

\scriptsize \noindent
%\footnotesize \noindent
 \FlightGear{} is a free flight simulator developed collectively over the
 Internet under the GPL.\\

 \noindent
 \textbf{Main Web site:} \web{http://www.flightgear.org/}
 \medskip

 \hspace*{-8mm}
\begin{tabular}{ll}
\textbf{Program Start:}  & Linux/UNIX via the script runfgfs under /FlightGear,\\
                         & Windows via the Batch file runfgfs.bat under /FlightGear
\end{tabular}
 \medskip

\hspace*{-8mm}
\begin{tabular}{ll}
\textbf{Engine Start:}  & Put ignition switch to ''BOTH''. Set mixture to 100\,\%. Set throttle to about 25\,\%.\\
                        & Operate starter using the SPACE key. Set throttle back to idle after starting the engine. Release parking brake, if applied.
\end{tabular}
\medskip

%%%%%%%%%%%%%%%%%%%%%%%%%%%%%%%%%%%%%%%%%%%%%%%%%%%%%%%%%%%%
 \noindent
 \textbf{Keyboard controls:}
\begin{multicols}{2}
 \noindent
 Tab.\,1: \textit{Main keyboard controls on the numeric keypad with
 activated \texttt{NumLock} key}.\\

\noindent
/home/tonghuix/git-pro/flightgear-getstart/bin/../source/tab2.tex
\bigskip

%%%%%%%%%%%%%%%%%%%%%%%%%%%%%%%%%%%%%%%%%%%%%%%%%%%%%%%%%%%%
 \noindent
 Tab.\,2: \textit{View directions accessible after de-activating \texttt{NumLock} on the numeric keypad.}
\medskip

 \noindent
 %%
%% tab3.tex -- Flight Gear documentation: Installation and Getting Started
%% Keyboard controls table 2/View directions
%%
%% Written by Michael Basler, started September 1998.
%%
%% Copyright (C) 2002 Michael Basler (pmb@epost.de)
%%
%%
%% This program is free software; you can redistribute it and/or
%% modify it under the terms of the GNU General Public License as
%% published by the Free Software Foundation; either version 2 of the
%% License, or (at your option) any later version.
%%
%% This program is distributed in the hope that it will be useful, but
%% WITHOUT ANY WARRANTY; without even the implied warranty of
%% MERCHANTABILITY or FITNESS FOR A PARTICULAR PURPOSE.  See the GNU
%% General Public License for more details.
%%
%% You should have received a copy of the GNU General Public License
%% along with this program; if not, write to the Free Software
%% Foundation, Inc., 675 Mass Ave, Cambridge, MA 02139, USA.
%%
%% $Id: tab2.tex,v 0.6 2002/09/09 michael
%% (Log is kept at end of this file)
%%%%%%%%%%%%%%%%%%%%%%%%%%%%%%%%%%%%%%%%%%%%%%%%%%%%%%%%%%%%%%%%%%%%%%%%%%%%%%%%%%%%%%%%%%%%%%%%
\begin{tabular}{|c|l|}\hline
 Numeric Key  &  View direction\index{view directions}\\\hline
    Shift-8 & Forward\\
    Shift-7 & Left/forward\\
    Shift-4 & Left\\
    Shift-1 & Left/back\\
    Shift-2 & Back\\
    Shift-3 & Right/back\\
    Shift-6 & Right\\
    Shift-9 & Right/forward\\\hline
\end{tabular}

%% revision 0.5 2002/02/15 michael
%% Initial revision
\bigskip

%%%%%%%%%%%%%%%%%%%%%%%%%%%%%%%%%%%%%%%%%%%%%%%%%%%%%%%%%%%%
 \noindent
 Tab.\,3: \textit{Display options.}
\medskip

 \noindent
%%
%% tab4.tex -- Flight Gear documentation: The FlightGear Manual
%% Keyboard controls table 3/Additional view options
%%
%% Written by Michael Basler, started September 1998.
%%
%% Copyright (C) 2002 Michael Basler
%%
%%
%% This program is free software; you can redistribute it and/or
%% modify it under the terms of the GNU General Public License as
%% published by the Free Software Foundation; either version 2 of the
%% License, or (at your option) any later version.
%%
%% This program is distributed in the hope that it will be useful, but
%% WITHOUT ANY WARRANTY; without even the implied warranty of
%% MERCHANTABILITY or FITNESS FOR A PARTICULAR PURPOSE.  See the GNU
%% General Public License for more details.
%%
%% You should have received a copy of the GNU General Public License
%% along with this program; if not, write to the Free Software
%% Foundation, Inc., 675 Mass Ave, Cambridge, MA 02139, USA.
%%
%% $Id: tab3.tex,v 0.6 2002/09/09 michael
%% (Log is kept at end of this file)
%%%%%%%%%%%%%%%%%%%%%%%%%%%%%%%%%%%%%%%%%%%%%%%%%%%%%%%%%%%%%%%%%%%%%%%%%%%%%%%%%%%%%%%%%%%%%%%%
\begin{tabular}{|l|l|}\hline
 Key              &         Action\\\hline
 P                &    Toggle \Index{instrument panel} on/off \\
 c                &    Toggle3D/2D cockpit \index{2D cockpit} (if both are available) \index{3D cockpit}\index{cockpit}\\
 S                &    Cycle panel style full/mini\\
 Shift-F5/F6      &    Shift the panel in y direction\\
 Shift-F7/F8      &    Shift the panel in x direction\\
 Shift-F3         &    Read a panel from a property list\\
 Ctrl-c           &    Toggle panel/cockpit hotspot visibility\\
 i/I              &    Minimize/maximize HUD              \\
 h/H              &    Change color  of HUD/toggle HUD off\\
                  &    forward/backward      \\   \hline
 x/X              &    Zoom in/out\\
 v/V              &    Cycle \Index{view modes} forth and back\\
 Ctrl-v           &    Reset \Index{view modes} to pilot view\\
 z/Z              &    Increase/Decrease visibility (fog) \\
 F10              &    Toggle menu on/off\\ \hline
 \end{tabular}

%% revision 0.5 2002/02/15 michael
%% Initial revision

\bigskip
\rule{0mm}{15mm}

%%%%%%%%%%%%%%%%%%%%%%%%%%%%%%%%%%%%%%%%%%%%%%%%%%%%%%%%%%%%
 \noindent
 Tab.\,4: \textit{Autopilot and related controls.}
\medskip

\noindent
/home/tonghuix/git-pro/flightgear-getstart/bin/../source/tab5.tex
\medskip

%%%%%%%%%%%%%%%%%%%%%%%%%%%%%%%%%%%%%%%%%%%%%%%%%%%%%%%%%%%%
 \noindent
Tab.\,5: \textit{Special action of keys, if autopilot is enabled.}
\medskip

 \noindent
%%
%% tab6.tex -- Flight Gear documentation: The FlightGear Manual
%% Keyboard controls table 5/key actions for autopilot enabled
%%
%% Written by Michael Basler, started September 1998.
%%
%% Copyright (C) 2002 Michael Basler
%%
%%
%% This program is free software; you can redistribute it and/or
%% modify it under the terms of the GNU General Public License as
%% published by the Free Software Foundation; either version 2 of the
%% License, or (at your option) any later version.
%%
%% This program is distributed in the hope that it will be useful, but
%% WITHOUT ANY WARRANTY; without even the implied warranty of
%% MERCHANTABILITY or FITNESS FOR A PARTICULAR PURPOSE.  See the GNU
%% General Public License for more details.
%%
%% You should have received a copy of the GNU General Public License
%% along with this program; if not, write to the Free Software
%% Foundation, Inc., 675 Mass Ave, Cambridge, MA 02139, USA.
%%
%% $Id: tab5.tex,v 0.6 2002/09/09 michael
%% (Log is kept at end of this file)
%%%%%%%%%%%%%%%%%%%%%%%%%%%%%%%%%%%%%%%%%%%%%%%%%%%%%%%%%%%%%%%%%%%%%%%%%%%%%%%%%%%%%%%%%%%%%%%%
\begin{tabular}{|l|l|}\hline
\iflanguage{english}{
 Key          & Action\\\hline
  p           & Pause simulator \index{pause}\\
  a / A       & Simulation speed up/slow down\\
  t / T       & Clock speed up/slow down       \\
  Ctrl-R      & Instant replay \\
  F3          & Save screen shot\\
  ESC         & Exit program\\\hline
}{}
\iflanguage{french}{
 Touche       & Action\\\hline
  p           & Mettre le simulateur en pause \index{pause}\\
  a / A       & Vitesse de simulation acc\'{e}l'{e}rer/ralentir\\
  t / T       & Vitesse de l'horloge acc\'{e}l'{e}rer/ralentir\\
  Ctrl-R      & Ralenti instantan\'{e}\\
  F3          & Sauvegarder la capture d'\'{e}cran\\
  ESC         & Quitter le programme\\\hline
}{}
 \end{tabular}

%% revision 0.5 2002/02/15 michael
%% Initial revision

\medskip


%%%%%%%%%%%%%%%%%%%%%%%%%%%%%%%%%%%%%%%%%%%%%%%%%%%%%%%%%%%%
 \noindent
Tab.\,6: \textit{Engine control keys}
\medskip

 \noindent
%%
%% tab7.tex -- Flight Gear documentation: Installation and Getting Started
%% Keyboard controls table 6/Engine related controls
%%
%% Written by Michael Basler, started September 1998.
%%
%% Copyright (C) 2002 Michael Basler (pmb@epost.de)
%%
%%
%% This program is free software; you can redistribute it and/or
%% modify it under the terms of the GNU General Public License as
%% published by the Free Software Foundation; either version 2 of the
%% License, or (at your option) any later version.
%%
%% This program is distributed in the hope that it will be useful, but
%% WITHOUT ANY WARRANTY; without even the implied warranty of
%% MERCHANTABILITY or FITNESS FOR A PARTICULAR PURPOSE.  See the GNU
%% General Public License for more details.
%%
%% You should have received a copy of the GNU General Public License
%% along with this program; if not, write to the Free Software
%% Foundation, Inc., 675 Mass Ave, Cambridge, MA 02139, USA.
%%
%% $Id: tab6.tex,v 0.5 2002/15/02 michael
%% (Log is kept at end of this file)
%%%%%%%%%%%%%%%%%%%%%%%%%%%%%%%%%%%%%%%%%%%%%%%%%%%%%%%%%%%%%%%%%%%%%%%%%%%%%%%%%%%%%%%%%%%%%%%%
\begin{tabular}{|l|l|}\hline
Key      &  Action\\ \hline
   SPACE & Fire starter on selected engine(s)\\
   !     & Select 1st engine\\
   @		 & Select 2nd engine\\
  \#     & Select 3rd engine\\
  \$     & Select 4th engine\\
  \{     & Decrease Magneto on Selected Engine\\
  \}     & Increase Magneto on Selected Engine\\
  $\sim$   & Select all Engines\\\hline
\end{tabular}

%% revision 0.5 2002/02/15 michael
%% Initial revision
\medskip

%%%%%%%%%%%%%%%%%%%%%%%%%%%%%%%%%%%%%%%%%%%%%%%%%%%%%%%%%%%%
 \noindent
Tab.\,7: \textit{Miscellaneous keyboard controls.}
\medskip

 \noindent
/home/tonghuix/git-pro/flightgear-getstart/bin/../source/tab8.tex
\medskip


\end{multicols}

 \noindent
 \textbf{Mouse controlled functions:}
 There are three mouse modi. In the usual mode (pointer curser) panel's controls can be operated
 with the mouse. To change a control, click with the left/middle mouse button on the
 corresponding knob/lever. While the left mouse button leads to small increments/decrements,
 the middle one makes greater ones. Clicking on the left hand side of the knob/lever
 decreases the value, while clicking on the right hand side increases it.

 \noindent
 Right clicking the mouse activates the simulator control mode (cross hair cursor). This allows
 control of aileron/elevator via the mouse in absence of a joystick
 (enable \texttt{-$ $-enable-auto-coordination} in this case).

 \noindent
 Right clicking the mouse another time activates the view control mode (arrow cursor).
 This allows changing direction of view via the mouse.

 \noindent
 Right clicking the mouse once more resets it into the initial state.
 \medskip

 \noindent
 Short Reference by M. Basler (pmb@epost.de) for \FlightGear{} version 0.8.0.\\
 Published under the GPL (\web{http://www.gnu.org/copyleft/gpl.html})

\end{document}

%% Revision 0.4  2001/01/05  michael
%% Initial revision based on getting Started 0.4
%% Revision 0.5 2002/02/15   michael
%% based on \inputs of revised table files from Getting Started 0.5
