%%
%% preface.tex -- Flight Gear documentation: The FlightGear Manual
%% Chapter file
%%
%% Written by Michael Basler, started September 1998.
%%
%% Copyright (C) 2002 Michael Basler
%%
%%
%% This program is free software; you can redistribute it and/or
%% modify it under the terms of the GNU General Public License as
%% published by the Free Software Foundation; either version 2 of the
%% License, or (at your option) any later version.
%%
%% This program is distributed in the hope that it will be useful, but
%% WITHOUT ANY WARRANTY; without even the implied warranty of
%% MERCHANTABILITY or FITNESS FOR A PARTICULAR PURPOSE.  See the GNU
%% General Public License for more details.
%%
%% You should have received a copy of the GNU General Public License
%% along with this program; if not, write to the Free Software
%% Foundation, Inc., 675 Mass Ave, Cambridge, MA 02139, USA.
%%
%% $Id: preface.tex,v 0.6 2002/09/09 michael
%% (Log is kept at end of this file)

%%%%%%%%%%%%%%%%%%%%%%%%%%%%%%%%%%%%%%%%%%%%%%%%%%%%%%%%%%%%%%%%%%%%%%%%%%%%%%%%%%%%%%%%%%%%%%%
\chapter*{Preface\label{preface}}
%%%%%%%%%%%%%%%%%%%%%%%%%%%%%%%%%%%%%%%%%%%%%%%%%%%%%%%%%%%%%%%%%%%%%%%%%%%%%%%%%%%%%%%%%%%%%%%

\FlightGear{} is a free Flight Simulator developed cooperatively over the Internet 
by a group of flight simulation and programming enthusiasts. "The
FlightGear Manual" is meant to give beginners a guide in getting
\FlightGear{} up and running, and themselves into the air. It is not
intended to provide complete documentation of all the features and
add-ons of \FlightGear{} but, instead, aims to give a new user the best
start to exploring what \FlightGear{} has to offer.

This version of the document was written for \FlightGear{} version 2.0.0.
Users of earlier versions of \FlightGear{} will still find this document
useful, but some of the features described may not be present.

This guide is split into three parts and is structured as follows.

\medskip

\noindent
\textbf{Part I: Installation}
\medskip

 \noindent
Chapter~\ref{free}, \textit{Want to have a free flight? Take \FlightGear{}}, introduces
\FlightGear{}, provides background on the philosophy behind it and describes the system requirements.
 \medskip

 \noindent
In Chapter~\ref{prefligh}, \textit{Preflight: Installing \FlightGear{}}, you will find
instructions for installing the binaries\index{binary distribution} and additional scenery and aircraft. 
 \medskip

\noindent
\textbf{Part II: Flying with \FlightGear{}}
\medskip

 \noindent
  The following Chapter~\ref{takeoff}, \textit{Takeoff: How to start
  the program}, describes how to actually start the installed program.
  It includes an overview on the numerous command line options as well
  as configuration files.
 \medskip

 \noindent
  Chapter~\ref{flight}, \textit{In-flight: All about instruments,
  keystrokes and menus}, describes how to operate the program, i.\,e\.
  how to actually fly with \FlightGear{}\hspace{-1mm}. This includes a
  (hopefully) complete list of pre-defined keyboard commands, an
  overview on the menu entries, detailed descriptions on the instrument
  panel and HUD (head up display), as well as hints on using the mouse
  functions.
 \medskip

 \noindent
  Chapter~\ref{features}, \textit{Features} describes some of the special
  features that \FlightGear{} offers to the advanced user.
 \medskip

\noindent
\textbf{Part III: Tutorials}
\medskip

 \noindent
 Chapter~\ref{tutorials}, \textit{Tutorials},
provides information on the many tutorials available for new pilots.
 \medskip

 \noindent
 Chapter~\ref{basic}, \textit{A Basic Flight Simulator Tutorial},
provides a tutorial on the basics of flying, illustrated with many
examples on how things actually look in \FlightGear{}.
 \medskip

 \noindent
 Chapter~\ref{crosscountry}, \textit{A Cross Country Flight Tutorial},
describes a simple cross-country flight in the San Fransisco area that
can be run with the default installation.
 \medskip

 \noindent
 Chapter~\ref{ifr}, \textit{An IFR Cross Country Flight Tutorial},
describes a similar cross-country flight making use of the instruments to 
successfully fly in the clouds under Instrument Flight Rules (IFR).
 \medskip

\noindent
\textbf{Appendices}
\medskip

 \noindent
  In Appendix~\ref{missed}, \textit{Missed approach: If anything refuses to work},
   we try to help you work through some common problems faced when using \FlightGear{}.
 \medskip

 \noindent
  In the final Appendix~\ref{landing}, \textit{Landing: Some further thoughts before leaving the plane}, we would like to give credit to those who deserve it, sketch an overview
on the development of \FlightGear and point out what remains to be done.
 \medskip

\section{Condensed Reading}

For those who don't want to read this document from cover to cover, we suggest reading the following
sections in order to provide an easy way to get into the air:

\begin{tabular}{ll}
 Installation :             &~\ref{prefligh}\\
 Starting the simulator :   &~\ref{takeoff}\\
 Using the simulator :      &~\ref{flight}\\
\end{tabular}
\bigskip

\section{Instructions For the Truly Impatient}

We know most people hate reading manuals. If you are sure the graphics driver for your card supports \Index{OpenGL} (check documentation; for instance in general \Index{NVIDIA} graphics cards do) and you are using Windows, Mac OS-X or Linux, you can probably skip at least Part I of this manual and use pre-compiled binaries\index{binaries!pre-compiled}. These as well as instructions on how to set them up, can be found at
 \medskip

\web{http://www.flightgear.org/Downloads/}.
 \medskip

 \noindent
If you are running Linux, you may find that \FlightGear{} is bundled with your distribution.

Once you have downloaded and installed the binaries, see Chapter \ref{takeoff} for details on starting the simulator.

\section{Further Reading}

\noindent
 While this introductory guide is meant to be self contained, we strongly suggest having a look into further documentation, especially in case of trouble:

\begin{itemize}
 \item A handy \textbf{leaflet}\index{leaflet} on operation for printout can be found in base package under \texttt{/FlightGear/Docs}, and is also available from

\web{http://www.flightgear.org/Docs/FGShortRef.pdf}.
 
 \item Additional \textbf{user documentation} on particular features and function is available within the
base package under the directory \texttt{/FlightGear/Docs}.
 
 \item There is an official \FlightGear{} \textbf{wiki}\index{wiki} available at \web{http://wiki.flightgear.org}.
 \end{itemize}

%% Revision 0.00  1998/09/08  michael
%% Initial revision for version 0.41.
%% Revision 0.01  2002/01/01 michael
%% Included outline of the Guide, integrated former separate chapter ''Quickstart''
%% revision 0.5 2002/01/01 michael/martin
%% Hint on Linux distros
%% revision 0.6 2002/09/09 michael
%% minor corrections
%% revision 0.7 2005/11.10 stuart
%% Changes for moving compilation information to appendix
